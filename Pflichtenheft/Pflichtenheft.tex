\documentclass[parskip=full,11pt]{scrartcl}
%\usepackage{pdfpages}
\usepackage[utf8]{inputenc}
\usepackage[T1]{fontenc}
\usepackage[german]{babel}
\usepackage[yyyymmdd]{datetime} 
\usepackage{hyperref}
\usepackage[toc, nonumberlist]{glossaries}
\usepackage{csquotes}
\hypersetup{
		pdftitle={Pflichtenheft},
		bookmarks=true,
}
\usepackage{fancyhdr}%<-------------to control headers and footers
\usepackage[a4paper,margin=1in,footskip=.25in]{geometry}
\fancyhf{}
\fancyfoot[C]{\thepage} %<----to get page number below text
\pagestyle{fancy} %<-------the page style itself

\title{Pflichtenheft}
\subtitle{Autorisierungsmanagement für eine virtuelle Forschungsumgebung für Geodaten}
\author{Alex\\Anastasia\\Atanas\\Dannie\\ Houra\\Sonya\\}
\date{26.11.17}

% define custom lists
\usepackage{enumitem}

% add glossary
\usepackage{glossaries}
\makeglossaries
\def\threedigits#1{%
  \ifnum#1<100 0\fi
  \ifnum#1<10 0\fi
  \number#1}
\begin{document}

\begin{titlepage}
	
	\begin{center}
	{\scshape\LARGE\bfseries Pflichtenheft \par}
	\vspace{1cm}
	{\scshape\Large Praxis der Softwareentwicklung\\}
	\vspace{1cm}
	{\scshape\Large Wintersemester 17/18\\}
	\vspace{3cm}
	{\huge\bfseries Autorisierungsmanagement für eine virtuelle Forschungsumgebung für Geodaten\par}
	\vspace{2cm}
	\vfill
	{\bfseries {\Large Autoren}:\par}
	{\Large Alex}\\%TODO Nachname ergaenzen
	{\Large Anastasia}\\%TODO Nachname ergaenzen
	{\Large Atanas}\\%TODO Nachname ergaenzen
	{\Large Dannie}\\%TODO Nachname ergaenzen
	{\Large Houraalsadat Mortazavi Moshkenan}\\
	{\Large Sonya}\\%TODO Nachname ergaenzen
	\vfill
	{\large 26.11.17 \par}
	\end{center}
\end{titlepage}
\tableofcontents

\newpage
%Eineitung?
\section{Zielbestimmung}
Das Produkt dient  zum Autoriesierungsmanagement des  \gls{V-For-WaTer} Web-Portals.Dadurch werden die jenigen in dem Webportal registierten \gls{Benutzer} in die Lage versetzt, ihre Profil zu verwalten,Ressourcen zu bearbeiten und zu zugreiffen.

\subsection{Musskriterien}
\subsection*{Benuzter}
\begin{itemize}[itemsep=0pt]
 \item bla
\end{itemize}

\subsection*{Die Gruppen}
\begin{itemize}[itemsep=0pt]
 \item bla
\end{itemize}

\subsection*{Administrator der Gruppe}
\begin{itemize}[itemsep=0pt]
 \item 
\end{itemize}




\subsection{Wunschkriterien}
\begin{itemize}[itemsep=0pt]
\item bla
\end{itemize}
\subsection{Abgerenzungskriterien}
\begin{itemize}[itemsep=0pt]
\item bla
\end{itemize}


\section{Produkteinsatz}

\subsection{Anwendungsbereiche}
\subsection{Zielgruppen}
\subsection{Betriebsbedingungen}


\section{Produktumgebung}
\subsection{Sofware}
\subsection{Hardware}


\section{Funktionale Anforderungen}
\subsection{Benutzerkontofunktionen}
\begin{enumerate}[label={\textbf{/F\protect\threedigits{\theenumi}}}, leftmargin=*]
\item \textit{bla}
\end{enumerate}

\subsubsection{Administratorfunktionen}
\begin{enumerate}[label={\textbf{/F\protect\threedigits{\theenumi}}}, leftmargin=*]
\item \textit{bla}
\end{enumerate}

\section{Produktdaten}
\begin{enumerate}[label={\textbf{/D\protect\threedigits{\theenumi}}}, leftmargin=*]
\item \textit{bla}
\end{enumerate}
\section{Nichtfunktionale Anforderungen}
\begin{enumerate}[label={\textbf{/NF\protect\threedigits{\theenumi}}}, leftmargin=*]
\item \textit{bla}
\end{enumerate}



\section{Benutzungsschnittstelle}

\section{Qualitätsbestimmungen}

\section{Globale Testfälle und Testszenarien}

\newpage
\printglossary	
\end{document}
\grid
\grid
