\documentclass[parskip=full,11pt]{scrartcl}
%\usepackage{pdfpages}
\usepackage[utf8]{inputenc}
\usepackage{amssymb}
\usepackage[T1]{fontenc}
\usepackage[german]{babel}
\usepackage[yyyymmdd]{datetime} 
\usepackage{hyperref}
\usepackage[toc, nonumberlist, automake]{} %added automake option
\usepackage{csquotes}
\usepackage{graphicx}
\usepackage{longtable}
\usepackage{multirow}
\usepackage{pbox}
\hypersetup{
 		pdftitle={VForWater-Implementierung},
 }
\usepackage{fancyhdr}%<-------------to control headers and footers
\usepackage[a4paper,margin=1in,footskip=.25in]{geometry}
\fancyhf{}
\fancyfoot[C]{\thepage} %<----to get page number below text
\pagestyle{fancy} %<-------the page style itself
 
\title{Qualitätssicherung}
\subtitle{Autorisierungsmanagement für eine virtuelle Forschungsumgebung für Geodaten}
\author{Alex\\Anastasia\\Atanas\\Dannie\\ Houra\\Sonya\\}
\date{11.01.18}
 % define custom lists
\usepackage{enumitem}


\usepackage{linegoal,listings}
\newsavebox{\mylisting}
\makeatletter
\newcommand{\lstInline}[2][,]{%
	\begingroup%
	\lstset{#1}% Set any keys locally
	\begin{lrbox}{\mylisting}\lstinline!#2!\end{lrbox}% Store listing in \mylisting
	\setlength{\@tempdima}{\linegoal}% Space left on line.
	\ifdim\wd\mylisting>\@tempdima\hfill\\\fi% Insert line break
	\lstinline!#2!% Reset listing
	\endgroup%
}
\makeatother
\setlength{\parindent}{0pt}% Just for this example

\lstset{basicstyle=\footnotesize\ttfamily,breaklines=true}
\lstset{framextopmargin=50pt,frame=bottomline,showstringspaces=false,upquote=true}


 
\begin{document}
 
 \begin{titlepage}
 	
 	\begin{center}
 	\includegraphics[width=0.5\linewidth]{res/KITLogo.png}\\
 	\vspace{2cm}
 	{\scshape\LARGE\bfseries Qualitätssicherung \par}
 	\vspace{0.5cm}
 	{\scshape\Large Praxis der Softwareentwicklung\\}
 	\vspace{1cm}
 	{\scshape\Large Wintersemester 17/18\\}
 	\vspace{2cm}
 	{\huge\bfseries Autorisierungsmanagement für eine virtuelle Forschungsumgebung für Geodaten\par}
 	\vspace{2cm}
 	\vfill
 	{\bfseries {\Large Autoren}:\par}
 	{\Large Bachvarov, Aleksandar }\\
 	{\Large Dimitrov, Atanas }\\
 	{\Large Mortazavi Moshkenan, Houraalsadat }\\
 	{\Large Sakly, Khalil }\\
 	{\Large Slobodyanik, Anastasia }\\
 	{\Large Voneva, Sonya}\\
 	\vfill
 	{\large 14.03.18 \par}
 	\end{center}
 \end{titlepage}
 
 \tableofcontents
 \newpage
 \section{Einleitung}
\begin{center}
\textit{,,Testing shows the presence of bugs, not their absence``}
\end{center}
\begin{flushright}
- Edsger W. Dijkstra\\
\end{flushright}
Das Testen in der Software-Entwicklung ist das Prozess der Untersuchung des Produktes auf Unterschiede zwischen dem beobachteten Verhalten und dem erwarteten Verhalten des Systemmodells.
In dem Wasserfall-Vorgehensmodell wird die Testphase direkt vor der Auslieferung des Produkts durchgeführt und wird als Sicherstellung festgelegter Qualitätsanforderungen betrachtet.\\
Testphase ist 
 \newpage
 \section{Qualität}
 Wie wird Qualität gemessen.
 
 \newpage
\section{Komponenten-Test}
Views unit Tests

\begin{longtable}[c]{|p{0.4cm}|p{2.5cm}|p{3.5cm}|p{5cm}|p{2cm}|p{1.8cm}|}
\caption{My caption}
\label{my-label}\\
\hline
\textbf{\#} & \textbf{Testcase}&\textbf{Methoden}& \textbf{Beschreibung} & \textbf{Erwartung} & \textbf{Ergebnis} \\ \hline
\endfirsthead
%
\endhead
%
\multirow{2}{*}{}1 & \multirow{2}{*}{} TestHomeView & test-not-logged-in & Getestet, ob die Homeseite gezeigt wird, wenn man nicht eingeloggt ist. Das wird durch den zurückgegebene HTTP-Statuscode gemacht.& 302  & 302 \\ \cline{3-6} &   & test-normal & Getestet, ob die Homeseite gezeigt wird, wenn man eingeloggt ist.  & 200 & 200 \\ \hline
\multirow{3}{*}{} 2& \multirow{3}{*}{} TestResourceManager&  test-not-logged-in & Getestet, ob die ResourceManagerseite gezeigt wird, wenn man nicht eingeloggt ist. & 302  & 302  \\ \cline{3-6} & & test-logged-in-no-admin & Getestet, ob die ResourceManagerseite gezeigt wird, wenn man eingeloggt ist, aber zwar nicht als Admin. & 302 & 302 \\ \cline{3-6} & & test-normal & Getestet, ob die ResourceManagerseite gezeigt wird, wenn man als Admin  eingeloggt ist. & 200 & 200  \\ \hline
\multirow{3}{*}{} 3& \multirow{3}{*}{} TestUserManager&  test-not-logged-in & Getestet, ob die UserManagerseite gezeigt wird, wenn man nicht eingeloggt ist. & 302  & 302  \\ \cline{3-6} & & test-logged-in-no-admin & Getestet, ob die UserManagerseite gezeigt wird, wenn man eingeloggt ist, aber zwar nicht als Admin. & 302 & 302 \\ \cline{3-6} & & test-normal & Getestet, ob die UserManagerseite gezeigt wird, wenn man als Admin  eingeloggt ist. & 200 & 200  \\ \hline
\multirow{5}{*}{} 4& \multirow{5}{*}{} TestProfileView& test-not-logged-in & Getestet, ob die Profileseite gezeigt wird, wenn man nicht eingeloggt ist.& 302  & 302 \\ \cline{3-6} &   & test-normal & Getestet, ob die Profileseite gezeigt wird, wenn man eingeloggt ist.  & 200 & 200 \\ \cline{3-6} 
                  &                   & test-pagination-user & Getestet, ob der Benutzer die zwei Zugriffsrequesten sieht, die ihm gesendet wurden. & 2 & 2 \\ \cline{3-6} 
                  &                   & test-pagination-admin-page-1 & Getestet, ob der Admin  vier von den fünf Requesten, die ihm gesendet wurden,  auf der ersten Seite sieht. & 2 & 2 \\ \cline{3-6} 
                  &                   & test-pagination-admin-page-2 & Getestet, ob der Admin  das Fünfte von den fünf Requesten, die ihm gesendet wurden,  auf der zweiten Seite sieht.& 2 & 2 \\  \hline
\multirow{3}{*}{} 5& \multirow{3}{*}{} TestMyResourcesView&  &  &  &  \\ \cline{3-6} 
                  &                   &  &  &  &  \\ \cline{3-6} 
                  &                   &  &  &  &  \\ \hline
\multirow{4}{*}{}6 & \multirow{4}{*}{} TestSendDeletionRequest&  &  &  &  \\ \cline{3-6} 
                  &                   &  &  &  &  \\ \cline{3-6} 
                  &                   &  &  &  &  \\ \cline{3-6} 
                  &                   &  &  &  &  \\ \hline
\multirow{4}{*}{}7& \multirow{4}{*}{} TestCancelDeletionRequest&  &  &  &  \\ \cline{3-6} 
                  &                   &  &  &  &  \\ \cline{3-6} 
                  &                   &  &  &  &  \\ \cline{3-6} 
                  &                   &  &  &  &  \\ \hline
\multirow{3}{*}{}8 & \multirow{3}{*}{} TestApproveAccesRequest&  &  &  &  \\ \cline{3-6} 
                  &                   &  &  &  &  \\ \cline{3-6} 
                  &                   &  &  &  &  \\ \hline
\multirow{3}{*}{} 9& \multirow{3}{*}{} TestDenyAccesRequest&  &  &  &  \\ \cline{3-6} 
                  &                   &  &  &  &  \\ \cline{3-6} 
                  &                   &  &  &  &  \\ \hline
\multirow{4}{*}{} 10& \multirow{4}{*}{} TestSendAccessRequest&  &  &  &  \\ \cline{3-6} 
                  &                   &  &  &  &  \\ \cline{3-6} 
                  &                   &  &  &  &  \\ \cline{3-6} 
                  &                   &  &  &  &  \\ \hline
\multirow{4}{*}{} 11& \multirow{4}{*}{} TestCancelAccessRequest&  &  &  &  \\ \cline{3-6} 
                  &                   &  &  &  &  \\ \cline{3-6} 
                  &                   &  &  &  &  \\ \cline{3-6} 
                  &                   &  &  &  &  \\ \hline
\multirow{3}{*}{} 12& \multirow{3}{*}{} TestDeleteResourceView&  &  &  &  \\ \cline{3-6} 
                  &                   &  &  &  &  \\ \cline{3-6} 
                  &                   &  &  &  &  \\ \hline
\multirow{2}{*}{} 13& \multirow{2}{*}{} TestDeleteResourceView&  &  &  &  \\ \cline{3-6} 
                  &                   &  &  &  &  \\ \hline
\multirow{3}{*}{} 14& \multirow{3}{*}{} TestEditNameView&  &  &  &  \\ \cline{3-6} 
                  &                   &  &  &  &  \\ \cline{3-6} 
                  &                   &  &  &  &  \\ \hline
\multirow{4}{*}{}15 & \multirow{4}{*}{} TestResourcesOverview&  &  &  &  \\ \cline{3-6} 
                  &                   &  &  &  &  \\ \cline{3-6} 
                  &                   &  &  &  &  \\ \cline{3-6} 
                  &                   &  &  &  &  \\ \hline
\multirow{5}{*}{}16& \multirow{5}{*}{} TestResourcesOverviewSearch&  &  &  &  \\ \cline{3-6} 
                  &                   &  &  &  &  \\ \cline{3-6} 
                  &                   &  &  &  &  \\ \cline{3-6} 
                  &                   &  &  &  &  \\ \cline{3-6} 
                  &                   &  &  &  &  \\ \hline
\multirow{7}{*}{} 17& \multirow{7}{*}{} TestPermissionEditingView&  &  &  &  \\ \cline{3-6} 
                  &                   &  &  &  &  \\ \cline{3-6} 
                  &                   &  &  &  &  \\ \cline{3-6} 
                  &                   &  &  &  &  \\ \cline{3-6} 
                  &                   &  &  &  &  \\ \cline{3-6} 
                  &                   &  &  &  &  \\ \cline{3-6} 
                  &                   &  &  &  &  \\ \hline
\end{longtable}
\newpage


\section{Integration-Test}
Test des Zusammenspiels der Komponenten.

\newpage
\section{System-Test}
System-Test im Sinne von Software Entwicklung ist der abschließende Test, der vom Entwickler durchgeführt wird. Das Ziel ist das Verhalten des Produkts als Ganzes in einer realen Umgebung (immer noch ohne Kunden) zu beobachten.\\
Im Rahmen unseres Projekts sind die System-Tests als manuelle Tests bezeichnet. Sie entsprechen möglichen Szenarien, die durch Interaktion mit dem System entstehen. Für die klare Unterscheidung zwischen den verschiedenen Rollen, die man im Portal haben kann, sind die manuelle Tests in drei Gruppen aufgeteilt - für Benutzer, für Besitzer und für Administrator.

\newpage
\section{Zusammenfassung}
Testüberdeckung. Fazit über die Qualität des Produkts.
\end{document}
\grid
