\documentclass[parskip=full,11pt]{scrartcl}
%\usepackage{pdfpages}
\usepackage[utf8]{inputenc}
\usepackage{amssymb}
\usepackage[T1]{fontenc}
\usepackage[german]{babel}
\usepackage[yyyymmdd]{datetime} 
\usepackage{hyperref}
\usepackage[toc, nonumberlist, automake]{} %added automake option
\usepackage{csquotes}
\usepackage{graphicx}
\usepackage{longtable}
\usepackage{multirow}
\usepackage{pbox}
\usepackage{array}
\newcolumntype{P}[1]{>{\centering\arraybackslash}p{#1}}
\hypersetup{
 		pdftitle={VForWater-Implementierung},
 }
\usepackage{fancyhdr}%<-------------to control headers and footers
\usepackage[a4paper,margin=1in,footskip=.25in]{geometry}
\fancyhf{}
\fancyfoot[C]{\thepage} %<----to get page number below text
\pagestyle{fancy} %<-------the page style itself
 
\title{Qualitätssicherung}
\subtitle{Autorisierungsmanagement für eine virtuelle Forschungsumgebung für Geodaten}
\author{Alex\\Anastasia\\Atanas\\Dannie\\ Houra\\Sonya\\}
\date{11.01.18}
 % define custom lists
\usepackage{enumitem}


\usepackage{linegoal,listings}
\newsavebox{\mylisting}
\makeatletter
\newcommand{\lstInline}[2][,]{%
	\begingroup%
	\lstset{#1}% Set any keys locally
	\begin{lrbox}{\mylisting}\lstinline!#2!\end{lrbox}% Store listing in \mylisting
	\setlength{\@tempdima}{\linegoal}% Space left on line.
	\ifdim\wd\mylisting>\@tempdima\hfill\\\fi% Insert line break
	\lstinline!#2!% Reset listing
	\endgroup%
}
\makeatother
\setlength{\parindent}{0pt}% Just for this example

\lstset{basicstyle=\footnotesize\ttfamily,breaklines=true}
\lstset{framextopmargin=50pt,frame=bottomline,showstringspaces=false,upquote=true}


 
\begin{document}
 
 \begin{titlepage}
 	
 	\begin{center}
 	\includegraphics[width=0.5\linewidth]{res/KITLogo.png}\\
 	\vspace{2cm}
 	{\scshape\LARGE\bfseries Qualitätssicherung \par}
 	\vspace{0.5cm}
 	{\scshape\Large Praxis der Softwareentwicklung\\}
 	\vspace{1cm}
 	{\scshape\Large Wintersemester 17/18\\}
 	\vspace{2cm}
 	{\huge\bfseries Autorisierungsmanagement für eine virtuelle Forschungsumgebung für Geodaten\par}
 	\vspace{2cm}
 	\vfill
 	{\bfseries {\Large Autoren}:\par}
 	{\Large Bachvarov, Aleksandar }\\
 	{\Large Dimitrov, Atanas }\\
 	{\Large Mortazavi Moshkenan, Houraalsadat }\\
 	{\Large Sakly, Khalil }\\
 	{\Large Slobodyanik, Anastasia }\\
 	{\Large Voneva, Sonya}\\
 	\vfill
 	{\large 14.03.18 \par}
 	\end{center}
 \end{titlepage}
 
 \tableofcontents
 \newpage
 \section{Einleitung}
\begin{center}
\textit{,,Testing shows the presence of bugs, not their absence``}
\end{center}
\begin{flushright}
- Edsger W. Dijkstra\\
\end{flushright}
Das Testen in der Software-Entwicklung ist der Prozess der Untersuchung des Produktes auf Unterschiede zwischen seinem beobachteten Verhalten und dem erwarteten Verhalten des modellierten Systems.
In dem Wasserfall-Vorgehensmodell wird die Testphase direkt vor der Auslieferung des Produkts durchgeführt und wird als Sicherstellung festgelegter Qualitätsanforderungen betrachtet.\\
Testphase ist mit anderen Entwicklungsphasen eng verbunden. Die erste Tests werden schon in der Implentierungsphase erstellt: Unit-Tests zu jedem View entsprechen dem Komponenten-Test, dessen Ziel ist, Fehler in jedem Teil des Systems aufzufinden. Diese Tests werden im Kapitel \hyperref[komponententest]{\textit{Komponenten-Test}} beschrieben.\\
Das Zusammenspiel einzelnen Komponenten wird im Integration-Test geprüft. In dieser Phase wird Rückblick auf Artefakt des Entwurfs genommen, um die Integrationsstrategie während dem Test zu berücksichtigen. Tests, die diese Phase umschließt, beschreiben wir im Kapitel \hyperref[integrationtest]{\textit{Integration-Test}}.\\
Der abschließende Teil der Testphase ist System-Test, der alle Komponenten zusammen prüft. In unserem Projekt werden dafür manuelle Tests angewendet, die ermöglichen, Benutzerszenarien nah zu Realität durchzuarbeiten. Testcases (Testszenarien) in dieser Phase werden anhand funktioneller Anforderungen aus dem Pflichtenheft erstellt und im Kapitel \hyperref[systemtest]{System-Test} beschrieben.\\
In der \hyperref[zusammenfassung]{Zusammenfassung} werden Ergebnisse der Testphase betrachtet: Codeüberdeckung vor und nach der Phase, behobene Fehler und unsere Beurteilung der Lieferfertigkeit des Produktes.
 \newpage
\section{Komponenten-Test} \label{komponententest}
Um die Einzelteile unseres Projekts auf ihre Korrektheit zu überprüfen, wurden sowohl die View-Klassen als auch ihre relevante URLs durch entsprechende Unittest-Klassen getestet. Für jede Test-Klasse werden alle mögliche Fälle durch verschiedene Methoden berücksichtigt.\\
In der vorliegenden Tabelle wird beschrieben, welche Test-Klassen implementiert wurden, was in den jeweiligen Tests geprüft wird und welches Ergebnis erwartet wird.

\begin{longtable}[c]{|p{0.4cm}|P{2.5cm}|p{3.5cm}|p{5cm}|p{3cm}|}
\caption{My caption}
\label{my-label}\\
\hline
\textbf{\#} & \textbf{Testcase}&\textbf{Methoden}& \textbf{Beschreibung} & \textbf{Ergebnis} \\ \hline
\endfirsthead
%
\endhead
%
\multirow{2}{*}{}1 & \multirow{2}{*}{} TestHomeView & test-not-logged-in & Getestet, ob die Homeseite gezeigt wird, wenn man nicht eingeloggt ist. Das wird durch den zurückgegebene HTTP-Statuscode gemacht.& HTTP Status Code 302: Der Benutzer wird zur Loginseite weitergeleitet  \\ \cline{3-5} &   & test-normal & Getestet, ob die Homeseite gezeigt wird, wenn man eingeloggt ist.  & HTPP Status Code 200: OK \\ \hline

\multirow{3}{*}{} 2& \multirow{3}{*}{} TestResourceManager&  test-not-logged-in & Getestet, ob die ResourceManagerseite gezeigt wird, wenn man nicht eingeloggt ist. & HTTP Status Code 302: Der Benutzer wird zur Loginseite weitergeleitet \\ \cline{3-5} & & test-logged-in-no-admin & Getestet, ob die ResourceManagerseite gezeigt wird, wenn man eingeloggt ist, aber zwar nicht als Admin. & HTTP Status Code 302: Der Benutzer wird zur Adminloginseite weitergeleitet \\ \cline{3-5} & & test-normal & Getestet, ob die ResourceManagerseite gezeigt wird, wenn man als Admin  eingeloggt ist. & HTPP Status Code 200: OK  \\ \hline

\multirow{3}{*}{} 3& \multirow{3}{*}{} TestUserManager&  test-not-logged-in & Getestet, ob die UserManagerseite gezeigt wird, wenn man nicht eingeloggt ist. & HTTP Status Code 302: Der Benutzer wird zur Loginseite weitergeleitet  \\ \cline{3-5} & & test-logged-in-no-admin & Getestet, ob die UserManagerseite gezeigt wird, wenn man eingeloggt ist, aber zwar nicht als Admin. & HTTP Status Code 302: Der Benutzer wird zur Adminloginseite weitergeleitet \\ \cline{3-5} & & test-normal & Getestet, ob die UserManagerseite gezeigt wird, wenn man als Admin  eingeloggt ist. & HTPP Status Code 200: OK \\ \hline

\multirow{5}{*}{} 4& \multirow{5}{*}{} TestProfileView& test-not-logged-in & Getestet, ob die Profileseite gezeigt wird, wenn man nicht eingeloggt ist.& HTTP Status Code 302: Der Benutzer wird zur Loginseite weitergeleitet  \\ \cline{3-5} &   & test-normal & Getestet, ob die Profileseite gezeigt wird, wenn man eingeloggt ist.  & HTPP Status Code 200: OK \\ \cline{3-5} 
                  &                   & test-pagination-user & Getestet, ob der Benutzer die zwei Zugriffsrequesten sieht, die ihm gesendet wurden. & Anzahl der gezeigten Requeste auf der Seite - 2 \\ \cline{3-5} 
                  &                   & test-pagination-admin-page-1 & Getestet, ob der Admin  vier von den fünf Requesten, die ihm gesendet wurden,  auf der ersten Seite sieht. & Anzahl der gezeigten Requeste auf der ersten Seite - 4 \\ \cline{3-5} 
                  &                   & test-pagination-admin-page-2 & Getestet, ob der Admin  das Fünfte von den fünf Requesten, die ihm gesendet wurden,  auf der zweiten Seite sieht.& Anzahl der gezeigten Requeste auf der zweiten Seite - 1 \\  \hline
   
                  
\multirow{3}{*}{} 5& \multirow{3}{*}{} TestMyResourcesView& test-not-logged-in & Getestet, ob die MyResourcesseite gezeigt wird, wenn man nicht eingeloggt ist.& HTTP Status Code 302: Der Benutzer wird zur Loginseite weitergeleitet \\ \cline{3-5} &   & test-normal & Getestet, ob die MyResourcesseite gezeigt wird, wenn man eingeloggt ist.  & HTPP Status Code 200: OK \\ \cline{3-5}& & test-resources-shown & Getestet, ob der Benutzer seine zwei Ressourcen auf der MyResourcesseite  sieht. &  Anzahl der gezeigten Ressourcen auf der Seite - 2  \\ \hline


\multirow{6}{*}{}6 & \multirow{6}{*}{} TestSendDeletionRequestView& test-not-logged-in & Getestet, ob man ein Löschrequest senden kann, wenn man nicht eingeloggt ist.& HTTP Status Code 302: Der Benutzer wird zur Loginseite weitergeleitet \\ \cline{3-5} 
                  &                   & test-not-existing-resource &  Getestet, man ein Löschrequest für eine Ressource senden kann, wenn die Ressource nicht existiert  &  HTTP Status Code 302: Der Benutzer wird zur MyResourcesseite weitergeleitet  \\ \cline{3-5} 
                  &                   & test-not-owner &Getestet, ob ein Benutzer, der die Ressource nicht besitzt, ein Löschrequest senden kann. & HTTP Status Code 403: Der Benutzer wird zur PermissionDeniedseite weitergeleitet    \\ \cline{3-5} 
                  &                   & test-staff-user &Getestet, ob der Admin ein Löschrequest senden kann.  &HTTP Status Code 302: Der Admin wird zur MyResourcesseite weitergeleitet \\ \cline{3-5} 
                  &                   & test-deletion-request-exists  & Getestet, was passiert, wenn ein Löschrequest für diese Ressource schon existiert.  & HTTP Status Code 302: Der Benutzer wird zur MyResourcesseite weitergeleitet   \\ \cline{3-5} 
                  &                  & test-normal  & Getestet, was es im normalen Fall passiert, wenn man eine Löschrequest sendet. & HTTP Status Code 302: Der Benutzer wird zur MyResourcesseite weitergeleitet   \\ \hline
                  
                  
\multirow{6}{*}{}7& \multirow{6}{*}{} TestCancelDeletionRequestView& test-not-logged-in & Getestet, ob man ein Löschrequest stornieren kann, wenn man nicht eingeloggt ist.& HTTP Status Code 302: Der Benutzer wird zur Loginseite weitergeleitet \\ \cline{3-5} 
                  &                   & test-not-existing-resource  & Getestet, ob man für eine nicht existierte Ressource einen Löschrequest zu stornieren versucht.  & HTTP Status Code 302: Der Benutzer wird zur MyResourcesseite weitergeleitet   \\ \cline{3-5} 
                  &                   & test-not-owner &Getestet, ob ein Benutzer, der die Ressource nicht besitzt, ein Löschrequest stornieren kann. & HTTP Status Code 403: Der Benutzer wird zur PermissionDeniedseite weitergeleitet   \\ \cline{3-5} 
                  &                   & test-staff-user &Getestet, ob der Admin ein Löschrequest ablehnen kann(Admin kann keine Löschrequeste senden).  &HTTP Status Code 302: Der Benutzer wird zur Loginseite weitergeleitet  \\ \cline{3-5} 
                  &                   & test-deletion-request-doesnt-exist  & Getestet, ob man einen nicht existierten Löschrequest zu stornieren versucht.  &  HTTP Status Code 302: Der Benutzer wird zur MyResourcesseite weitergeleitet  \\ \cline{3-5}
                  &                   & test-normal  & Getestet, was es im normalen Fall passiert, wenn man einen Löschrequest storniert. & HTTP Status Code 302: Der Benutzer wird zur MyResourcesseite weitergeleitet    \\ \hline
                  
                  
                  
\multirow{4}{*}{}8 & \multirow{4}{*}{} TestApproveAccessRequestView& test-not-logged-in & Getestet, ob man ein Zugriffsrequest genehmigen kann, wenn man nicht eingeloggt ist.& HTTP Status Code 302: Der Benutzer wird zur Loginseite weitergeleitet   \\ \cline{3-5}  
                  &                   & test-not-existing-request  & Getestet was passiert, wenn man einen nicht existierten Zugriffsrequest zu genehmigen versucht.  &  HTTP Status Code 302: Der Benutzer wird zur Profileseite weitergeleitet   \\ \cline{3-5}
                  &                   & test-not-owner &Getestet, ob ein Benutzer, der die Ressource nicht besitzt, ein Zugriffsrequest genehmigen kann. & HTTP Status Code 403: Der Benutzer wird zur PermissionDeniedseite weitergeleitet  \\ \cline{3-5}
                  &                   & test-normal  & Getestet, was es im normalen Fall passiert, wenn man einen Zugriffsrequest genehmigt. &   HTTP Status Code 302: Der Benutzer wird zur Profileseite weitergeleitet \\ \hline
                  
                  
                  
\multirow{4}{*}{} 9& \multirow{4}{*}{} TestDenyAccessRequestView& test-not-logged-in & Getestet, ob man ein Zugriffsrequest verweigern kann, wenn man nicht eingeloggt ist.& HTTP Status Code 302: Der Benutzer wird zur Loginseite weitergeleitet  \\ \cline{3-5}  
                  &                   & test-not-existing-request   & Getestet was passiert, wenn man einen nicht existierten Zugriffsrequest abzulehnen versucht.  &  HTTP Status Code 302: Der Benutzer wird zur Profileseite weitergeleitet    \\ \cline{3-5}
                  &                   & test-not-owner &Getestet, ob ein Benutzer, der die Ressource nicht besitzt, ein Zugriffsrequest verweigern kann. & HTTP Status Code 403: Der Benutzer wird zur PermissionDeniedseite weitergeleitet  \\ \cline{3-5}
                  &                   & test-normal  & Getestet, was es im normalen Fall passiert, wenn man einen Zugriffsrequest ablehnt. &   HTTP Status Code 302: Der Benutzer wird zur Profileseite weitergeleitet  \\ \hline
                  
                  
                  
\multirow{6}{*}{} 10& \multirow{6}{*}{} TestSendAccessRequestView& test-not-logged-in & Getestet, ob man ein Zugriffsrequest senden kann, wenn man nicht eingeloggt ist.& HTTP Status Code 302: Der Benutzer wird zur Loginseite weitergeleitet \\ \cline{3-5} 
                  &                   & test-not-existing-resource &  Getestet, man ein Zugriffsrequest für eine Ressource senden kann, wenn die Ressource nicht existiert  &  HTTP Status Code 302: Der Benutzer wird zur ResourcesOverviewseite weitergeleitet    \\ \cline{3-5}
				  &                   & test-reader &Getestet, ob ein Benutzer, der die Ressource schon zugreifen darf, ein Zugriffsrequest senden kann. & HTTP Status Code 302: Der Benutzer wird zur Loginseite weitergeleitet  \\ \cline{3-5}
                  &                   & test-staff-user &Getestet, ob der Admin ein Zugriffsrequest senden kann.  &HTTP Status Code 302: Der Benutzer wird zur Loginseite weitergeleitet  \\ \cline{3-5}  
                  &                   & test-access-request-exists  & Getestet, ob der Benutzer schon einen Zugriffsrequest gesendet hat.   &  HTTP Status Code 302: Der Benutzer wird zur ResourcesOverviewseite weitergeleitet    \\ \cline{3-5}
                  &                   &test-normal  & Getestet, was es im normalen Fall passiert, wenn man einen Zugriffsrequest sendet.  & HTTP Status Code 302: Der Benutzer wird zur ResourcesOverviewseite weitergeleitet    \\ \hline
                  
                  
                  
\multirow{6}{*}{} 11& \multirow{6}{*}{} TestCancelAccessRequestView& test-not-logged-in & Getestet, ob man ein Zugriffsrequest ablehnen kann, wenn man nicht eingeloggt ist.& HTTP Status Code 302: Der Benutzer wird zur Loginseite weitergeleitet \\ \cline{3-5} 
                 &                   & test-not-existing-resource  & Getestet, ob man für eine nicht existierte Ressource einen Zugriffsrequest zu stornieren versucht.  & HTTP Status Code 302: Der Benutzer wird zur ResourcesOverviewseite weitergeleitet   \\ \cline{3-5}
                  &                   & test-reader &Getestet, ob ein Benutzer, der die Ressource schon zugreifen darf, ein Zugriffsrequest stornieren kann. & HTTP Status Code 302: Der Benutzer wird zur Loginseite weitergeleitet \\ \cline{3-5} 
                  &                   & test-staff-user &Getestet, ob der Admin ein Zugriffsrequest ablehnen kann(Der Admin kann keine Zugriffsrequeste senden ).  &HTTP Status Code 302: Der Benutzer wird zur Loginseite weitergeleitet  \\ \cline{3-5} 
                  &                   & test-deletion-request-doesnt-exist  & Getestet, ob man einen nicht existierten Zugriffsrequest zu stornieren versucht.  &  HTTP Status Code 302: Der Benutzer wird zur ResourcesOverviewseite weitergeleitet    \\ \cline{3-5}

                  &                   &test-normal  & Getestet, was es im normalen Fall passiert, wenn man einen Zugriffsrequest storniert.  & HTTP Status Code 302: Der Benutzer wird zur ResourcesOverviewseite weitergeleitet    \\ \hline
                  
                  
                  
\multirow{4}{*}{} 12& \multirow{4}{*}{} TestDeleteResourceView& test-not-logged-in & Getestet, ob man eine Ressource löschen kann, wenn man nicht eingeloggt ist.& HTTP Status Code 302: Der Benutzer wird zur Loginseite weitergeleitet   \\ \cline{3-5} 
                  &                   & test-not-existing-resource  & Getestet, ob man für eine nicht existierte Ressource  löschen kann.  & HTTP Status Code 302: Der Benutzer wird zur MyResourcesseite weitergeleitet    \\ \cline{3-5} 
				  &                   & test-not-staff-user &Getestet, ob der Benutzer, der kein Admin ist, eine Ressource löschen kann.  &HTTP Status Code 302: Der Benutzer wird zur MyResourcesseite weitergeleitet     \\ \cline{3-5} 
                  &                   & test-normal  & Getestet, was es im normalen Fall passiert, wenn man eine Ressource löscht.  & HTTP Status Code 302: Der Benutzer wird zur MyResourcesseite weitergeleitet    \\ \hline
                  
                  
\multirow{2}{*}{} 13& \multirow{2}{*}{} TestEditNameView& test-not-logged-in & Getestet, ob man seinen Name ändern kann, wenn man nicht eingeloggt ist.& HTTP Status Code 302: Der Benutzer wird zur Loginseite weitergeleitet.    \\ \cline{3-5} 
                  &                   & test-normal  & Getestet, was es im normalen Fall passiert, wenn man seinen Name ändert.  & HTTP Status Code 302: Der Benutzer wird zur Profileseite weitergeleitet.    \\ \hline
                  
                  
\multirow{3}{*}{} 14& \multirow{3}{*}{} TestResourcesOverview& test-not-logged-in & Getestet, ob die ResourcesOverviewseite gezeigt ist, wen man nicht eingeloggt ist.& HTTP Status Code 302: Der Benutzer wird zur Loginseite weitergeleite   \\ \cline{3-5} 
                  &                   &test-normal  & Getestet, ob die ResourcesOverviewseite gezeigt ist, wenn man eingeloggt ist. &  HTPP Status Code 200: OK  \\ \cline{3-5} 
                  &                   &test-pagination-user  & Getestet, ob der Benutzer die vier existierten Ressourcen auf der Seite sieht. & Anzahl der gezeigten Ressourcen auf der seite-4   \\ \hline
                  
                  
\multirow{4}{*}{}15 & \multirow{4}{*}{} TestResourcesOverviewSearch& test-not-logged-in & Getestet, ob die ResourcesOverviewSearchseite gezeigt ist, wen man nicht eingeloggt ist. & HTTP Status Code 302: Der Benutzer wird zur Loginseite weitergeleite   \\ \cline{3-5} 
                  &                   &test-normal  & Getestet, ob die ResourcesOverviewSearchseite gezeigt ist, wenn man eingeloggt ist. &  HTPP Status Code 200: OK     \\ \cline{3-5} 
                  &                   &test-no-query  & Getestet was passiert, wenn der Benutzer keinen Zeichen für die Suchabfrage eingibt& HTTP Status Code 302: Der Benutzer wird zur ResourcesOverview weitergeleite   \\ \cline{3-5} 
                  &                   & test-valid-query & Getestet, ob der Benutzer die alle Ressourcen, deren Namen die eingegebene Zeichen für Suchabfrage enthalten, sieht.& Anzahl der gezeigten Ressourcen auf der Seite - 1    \\ \hline
                  
                  
\multirow{8}{*}{}16& \multirow{8}{*}{} TestPermissionEditingView& test-not-logged-in &  Getestet, ob man die EditPermissionsseite von einer Ressource sehen kann, wenn man nicht eingeloggt ist.& HTTP Status Code 302: Der Benutzer wird zur Loginseite weitergeleitet   \\ \cline{3-5}  
                  &                   & test-not-existing-resource-get &  Getestet, ob man die EditPermissionsseite von einer Ressource sehen kann, wenn die Ressource nicht existiert.  & HTTP Status Code 302: Der Benutzer wird zur Homeseite weitergeleitet  \\ \cline{3-5} 
                  &                   & test-not-existing-resource-post &  Getestet, ob man die Rechte für eine Ressource ändern kann, wenn die Ressource nicht existiert.  &  HTTP Status Code 302: Der Benutzer wird zur Homeseite weitergeleitet  \\ \cline{3-5} 
                  &  & test-not-authorized-user &Getestet, ob ein Benutzer die EditPermissionSeite von einer Ressource sehen kann, wenn man kein Besitzer der Ressource ist.  &  HTTP Status Code 403: Der Benutzer wird zur PermissionDeniedseite weitergeleitet  \\ \cline{3-5} 
                 &   & test-normal-get & Getestet, ob die EditPermissionseite im normalen Fall gezeigt wird.  & HTPP Status Code 200: OK   \\ \cline{3-5} 
                              &   & test-normal-post & Getestet, ob  man die Rechte für eine Ressource im normalen Fall ändern kann.  & HTTP Status Code 302: Der Benutzer wird zur MyResourcesseite weitergeleitet  \\ \cline{3-5}  
                  &                   & test-pagination-users & Getestet, ob die gezeigten Benutzer(Rechten von Benutzern) auf der Seite zwei sind. & Anzahl der gezeigten Benutzer auf der Seite - 2 \\ \hline
                  
                  
\multirow{8}{*}{} 17& \multirow{8}{*}{} TestPermissionEditingViewSearch& test-not-logged-in &  Getestet, ob man die EditPermissionsSearchseite von einer Ressource sehen kann, wenn man nicht eingeloggt ist.& HTTP Status Code 302: Der Benutzer wird zur Loginseite weitergeleitet   \\ \cline{3-5}  
                  &                   & test-not-existing-resource-get &  Getestet, ob man die EditPermissionsSearchseite von einer Ressource sehen kann, wenn die Ressource nicht existiert.  & HTTP Status Code 302: Der Benutzer wird zur Homeseite weitergeleitet  \\ \cline{3-5} 
                  &                   & test-not-existing-resource-post &  Getestet, ob man die Rechte für eine Ressource ändern kann, wenn die Ressource nicht existiert.  &  HTTP Status Code 302: Der Benutzer wird zur Homeseite weitergeleitet  \\ \cline{3-5} 
                  &  & test-not-authorized-user &Getestet, ob ein Benutzer die EditPermissionsSearchseite von einer Ressource sehen kann, wenn man kein Besitzer der Ressource ist.  &  HTTP Status Code 403: Der Benutzer wird zur PermissionDeniedseite weitergeleitet  \\ \cline{3-5} 
                  &                   &test-no-query  &  Getestet was passiert, wenn der Benutzer keinen Zeichen für die Suchabfrage eingibt& HTTP Status Code 302: Der Benutzer wird zur MyResourcesseite weitergeleite  \\ \cline{3-5}
                 &   & test-normal-get & Getestet, ob die EditPermissionsSearchseite im normalen Fall gezeigt wird.  & HTPP Status Code 200: OK   \\ \cline{3-5} 
                                &                   & test-valid-query & Getestet, ob man allen Benutzer(Rechten von Benutzern), deren Namen die eingegebenen Zeichen für Suchabfrage enthalten, sieht.& Anzahl der gezeigten Benutzer auf der Seite - 1    
       \\ \cline{3-5}                       &   & test-normal-post & Getestet, ob  man die Rechte für eine Ressource im normalen Fall ändern kann.  & HTTP Status Code 302: Der Benutzer wird zur MyResourcesseite weitergeleitet  \\ \hline 

                  
                  
\multirow{3}{*}{} 18& \multirow{7}{*}{} TestAddNewResourceView& test-not-logged-in & Getestet, ob die AddNewResourceseite gezeigt wird, wenn man nicht eingeloggt ist.& HTTP Status Code 302: Der Benutzer wird zur Loginseite weitergeleitet  \\ \cline{3-5} &   & test-normal & Getestet, ob die AddNewResourceseite gezeigt wird, wenn man eingeloggt ist.  & HTPP Status Code 200: OK \\ \hline 
                  &                   & test-no-resource-form & Getestet was passiert, wenn man eine Post-Anfrage ohne Resourcenlink sendet.   & HTTP Status Code 302: Der Benutzer wird zur MyResourcesseite weitergeleitet   \\ \hline
                  
                  
\multirow{4}{*}{} 19& \multirow{7}{*}{} TestOpenResourceView& test-not-logged-in & Getestet, ob die OpenResourceseite gezeigt wird, wenn man nicht eingeloggt ist. Das wird durch den zurückgegebene HTTP-Statuscode gemacht.& HTTP Status Code 302: Der Benutzer wird zur ResourcesOverviewseite weitergeleitet  \\ \cline{3-5}
 &                   & test-not-reader & Getestet, ob der Benutzer die OpenResourceseite sehen kann, wenn er keine Leserechte für die Ressource hat.  & HTTP Status Code 403: Der Benutzer wird zur PermissionDeniedseite weitergeleitet   \\ \cline{3-5} 
 &   & test-normal & Getestet, ob die OpenResourceseite im normalen Fall gezeigt wird.  & HTPP Status Code 200: OK \\ \hline
                  
                  
\multirow{4}{*}{}20 & \multirow{4}{*}{} TestApproveDeletionRequestView& test-not-logged-in & Getestet, ob man ein Löschrequest genehmigen kann, wenn man nicht eingeloggt ist.& HTTP Status Code 302: Der Benutzer wird zur Loginseite weitergeleitet   \\ \cline{3-5}  
                  &                   & test-not-existing-request  & Getestet was passiert, wenn man einen nicht existierten Löschrequest zu genehmigen versucht.  &  HTTP Status Code 302: Der Benutzer wird zur Profileseite weitergeleitet   \\ \cline{3-5}
                  &                   & test-not-admin &Getestet, ob ein Benutzer, der kein Admin ist, ein Löschrequest genehmigen kann. & HTTP Status Code 403: Der Benutzer wird zur PermissionDeniedseite weitergeleitet  \\ \cline{3-5}
                  &                   & test-normal  & Getestet, was es im normalen Fall passiert, wenn man einen Löschrequest genehmigt. &   HTTP Status Code 302: Der Benutzer wird zur Profileseite weitergeleitet \\ \hline
                  
                  
                  
\multirow{4}{*}{} 21& \multirow{4}{*}{} TestDenyDeletionRequestView& test-not-logged-in & Getestet, ob man ein Löschrequest verweigern kann, wenn man nicht eingeloggt ist.& HTTP Status Code 302: Der Benutzer wird zur Loginseite weitergeleitet  \\ \cline{3-5}  
                  &                   & test-not-existing-request   & Getestet was passiert, wenn man einen nicht existierten Löschrequest abzulehnen versucht.  &  HTTP Status Code 302: Der Benutzer wird zur Profileseite weitergeleitet    \\ \cline{3-5}
                  &                   & test-not-admin &Getestet, ob ein Benutzer, der kein Admin ist, ein Löschrequest verweigern kann. & HTTP Status Code 403: Der Benutzer wird zur PermissionDeniedseite weitergeleitet  \\ \cline{3-5}
                  &                   & test-normal  & Getestet, was es im normalen Fall passiert, wenn man einen Löschrequest ablehnt. &   HTTP Status Code 302: Der Benutzer wird zur Profileseite weitergeleitet  \\ \hline
                  
                  
\end{longtable}
\newpage


\section{Integration-Test} \label{integrationtest}
Test des Zusammenspiels der Komponenten.

\newpage
\section{System-Test} \label{systemtest}
System-Test im Sinne der Software-Entwicklung ist der abschließende Test, der vom Entwickler durchgeführt wird. Das Ziel dieses Tests ist das Verhalten des Produkts als Ganzes in einer realen Umgebung (immer noch ohne Kunden) zu beobachten.\\
Im Rahmen unseres Projekts sind die System-Tests als manuelle Tests bezeichnet. Sie entsprechen möglichen Szenarien, die durch Interaktion mit dem System entstehen. Als Basis dafür dienen die im Pflichtenheft vordefinierte globale Testfälle. Für die klare Unterscheidung zwischen den verschiedenen Rollen, die man im Portal haben kann, sind die manuelle Tests in drei Gruppen aufgeteilt - für Benutzer, für Besitzer und für Administrator.

\newpage
\section{Zusammenfassung} \label{zusammenfassung}
Testüberdeckung. Fazit über die Qualität des Produkts.
\end{document}
\grid
