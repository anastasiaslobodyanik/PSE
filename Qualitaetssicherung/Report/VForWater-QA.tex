\documentclass[parskip=full,11pt]{scrartcl}
%\usepackage{pdfpages}
\usepackage[utf8]{inputenc}
\usepackage{amssymb}
\usepackage[T1]{fontenc}
\usepackage[german]{babel}
\usepackage[yyyymmdd]{datetime} 
\usepackage{hyperref}
\usepackage[toc, nonumberlist, automake]{} %added automake option
\usepackage{csquotes}
\usepackage{graphicx}
\usepackage{longtable}
\usepackage{multirow}
\usepackage{pbox}
\usepackage{array}
\newcolumntype{P}[1]{>{\centering\arraybackslash}p{#1}}
\hypersetup{
 		pdftitle={VForWater-Implementierung},
 }
\usepackage{fancyhdr}%<-------------to control headers and footers
\usepackage[a4paper,margin=1in,footskip=.25in]{geometry}
\fancyhf{}
\fancyfoot[C]{\thepage} %<----to get page number below text
\pagestyle{fancy} %<-------the page style itself
 
\title{Qualitätssicherung}
\subtitle{Autorisierungsmanagement für eine virtuelle Forschungsumgebung für Geodaten}
\author{Alex\\Anastasia\\Atanas\\Dannie\\ Houra\\Sonya\\}
\date{11.01.18}
 % define custom lists
\usepackage{enumitem}


\usepackage{linegoal,listings}
\newsavebox{\mylisting}
\makeatletter
\newcommand{\lstInline}[2][,]{%
	\begingroup%
	\lstset{#1}% Set any keys locally
	\begin{lrbox}{\mylisting}\lstinline!#2!\end{lrbox}% Store listing in \mylisting
	\setlength{\@tempdima}{\linegoal}% Space left on line.
	\ifdim\wd\mylisting>\@tempdima\hfill\\\fi% Insert line break
	\lstinline!#2!% Reset listing
	\endgroup%
}
\makeatother
\setlength{\parindent}{0pt}% Just for this example

\lstset{basicstyle=\footnotesize\ttfamily,breaklines=true}
\lstset{framextopmargin=50pt,frame=bottomline,showstringspaces=false,upquote=true}


 
\begin{document}
 
 \begin{titlepage}
 	
 	\begin{center}
 	\includegraphics[width=0.5\linewidth]{res/KITLogo.png}\\
 	\vspace{2cm}
 	{\scshape\LARGE\bfseries Qualitätssicherung \par}
 	\vspace{0.5cm}
 	{\scshape\Large Praxis der Softwareentwicklung\\}
 	\vspace{1cm}
 	{\scshape\Large Wintersemester 17/18\\}
 	\vspace{2cm}
 	{\huge\bfseries Autorisierungsmanagement für eine virtuelle Forschungsumgebung für Geodaten\par}
 	\vspace{2cm}
 	\vfill
 	{\bfseries {\Large Autoren}:\par}
 	{\Large Bachvarov, Aleksandar }\\
 	{\Large Dimitrov, Atanas }\\
 	{\Large Mortazavi Moshkenan, Houraalsadat }\\
 	{\Large Sakly, Khalil }\\
 	{\Large Slobodyanik, Anastasia }\\
 	{\Large Voneva, Sonya}\\
 	\vfill
 	{\large 14.03.18 \par}
 	\end{center}
 \end{titlepage}
 
 \tableofcontents
 \newpage
 \section{Einleitung}
\begin{center}
\textit{,,Testing shows the presence of bugs, not their absence``}
\end{center}
\begin{flushright}
- Edsger W. Dijkstra\\
\end{flushright}
Das Testen in der Software-Entwicklung ist der Prozess der Untersuchung des Produktes auf Unterschiede zwischen seinem beobachteten Verhalten und dem erwarteten Verhalten des modellierten Systems.
In dem Wasserfall-Vorgehensmodell wird die Testphase direkt vor der Auslieferung des Produkts durchgeführt und als Sicherstellung festgelegter Qualitätsanforderungen betrachtet.\\
Die Testphase ist mit anderen Entwicklungsphasen eng verbunden. Die ersten Tests werden schon in der Implentierungsphase erstellt: Unittests zu jedem View entsprechen dem Komponenten-Test, dessen Ziel ist, Fehler in jedem Teil des Systems aufzufinden. Diese Tests werden im Kapitel \hyperref[komponententest]{\textit{Komponenten-Test}} beschrieben.\\
Das Zusammenspiel einzelner Komponenten wird im Integration-Test geprüft. In dieser Phase wird ein Rückblick auf das Artefakt des Entwurfs genommen, um die Integrationsstrategie während dem Test zu berücksichtigen. Tests, die diese Phase umschlie{\ss}t, beschreiben wir im Kapitel \hyperref[integrationtest]{\textit{Integration-Test}}.\\
Der abschlie{\ss}ende Teil der Testphase ist der Systemtest, der alle Komponenten zusammen prüft. In unserem Projekt werden dafür manuelle Tests angewendet, die ermöglichen, Benutzerszenarien realitätsnah durchzuarbeiten. Testcases (Testszenarien) in dieser Phase werden anhand funktioneller Anforderungen aus dem Pflichtenheft erstellt und im Kapitel \hyperref[systemtest]{SystemtTest} beschrieben.\\
Während der Testphase gefundene und behobene  Defekte werden im Kapitel \hyperref[bugs]{\textit{Aufgetretene Probleme}} erfasst.\\
In der \hyperref[zusammenfassung]{\textit{Zusammenfassung}} werden Ergebnisse der Testphase betrachtet: Codeüberdeckung vor und nach der Phase, behobene Fehler und unsere Beurteilung der Lieferfertigkeit des Produktes.
 \newpage
\section{Komponenten-Test} \label{komponententest}
Um die Einzelteile unseres Projekts auf ihre Korrektheit zu überprüfen, wurden sowohl die View-Klassen als auch ihre relevante URLs durch entsprechende Unittest-Klassen getestet. Für jede Test-Klasse werden alle mögliche Fälle durch verschiedene Methoden berücksichtigt.\\
In der vorliegenden Tabelle \ref{kom-tabelle} wird beschrieben, welche Test-Klassen implementiert wurden, was in den jeweiligen Tests geprüft wird und welches Ergebnis erwartet wird.

\begin{longtable}[c]{|p{0.4cm}|P{2.5cm}|p{3.5cm}|p{5cm}|p{3cm}|}
\caption{Komponenten-Test. Jede Test-Klasse (Testcase) testet die entsprechende View-Klasse, indem alle mögliche Fälle durch verschiedene Methoden berücksichtig werden.}
\label{kom-tabelle}\\
\hline
\textbf{\#} & \centerline{\textbf{Testcase}}&\centerline{\textbf{Methode}}& \centerline{\textbf{Beschreibung}} & \centerline{\textbf{Ergebnis}} \\ \hline
\endfirsthead
%
\endhead
%
\multirow{2}{*}{}1 & \multirow{2}{*}{} TestHomeView & test-not-logged-in & Getestet, ob die Homeseite gezeigt wird, wenn man nicht eingeloggt ist. Das wird durch den zurückgegebene HTTP-Statuscode gemacht.& HTTP Status Code 302: Der Benutzer wird zur ,,Login``-Seite weitergeleitet.  \\ \cline{3-5} &   & test-normal & Getestet, ob die Homeseite gezeigt wird, wenn man eingeloggt ist.  & HTPP Status Code 200: OK \\ \hline

\multirow{3}{*}{} 2& \multirow{3}{*}{} TestResourceManager&  test-not-logged-in & Getestet, ob die ,,Resource Manager``-Seite gezeigt wird, wenn man nicht eingeloggt ist. & HTTP Status Code 302: Der Benutzer wird zur ,,Login``-Seite weitergeleitet \\ \cline{3-5} & & test-logged-in-no-admin & Getestet, ob die ,,Resource Manager``-Seite gezeigt wird, wenn man zwar eingeloggt ist, aber nicht als Admin. & HTTP Status Code 302: Der Benutzer wird zur ,,Admin Login``-Seite weitergeleitet \\ \cline{3-5} & & test-normal & Getestet, ob die ,,Resource Manager``-Seite gezeigt wird, wenn man als Admin  eingeloggt ist. & HTPP Status Code 200: OK  \\ \hline

\multirow{3}{*}{} 3& \multirow{3}{*}{} TestUserManager&  test-not-logged-in & Getestet, ob die ,,User Manager``-Seite gezeigt wird, wenn man nicht eingeloggt ist. & HTTP Status Code 302: Der Benutzer wird zur ,,Login``-Seite weitergeleitet  \\ \cline{3-5} & & test-logged-in-no-admin & Getestet, ob die ,,User Manager``-Seite gezeigt wird, wenn man zwar eingeloggt ist, aber nicht als Admin. & HTTP Status Code 302: Der Benutzer wird zur ,,Admin Login``-Seite weitergeleitet \\ \cline{3-5} & & test-normal & Getestet, ob die ,,User Manager``-Seite gezeigt wird, wenn man als Admin  eingeloggt ist. & HTPP Status Code 200: OK \\ \hline

\multirow{5}{*}{} 4& \multirow{5}{*}{} TestProfileView& test-not-logged-in & Getestet, ob die ,,Profile``-Seite gezeigt wird, wenn man nicht eingeloggt ist.& HTTP Status Code 302: Der Benutzer wird zur ,,Login``-Seite weitergeleitet  \\ \cline{3-5} &   & test-normal & Getestet, ob die Profileseite gezeigt wird, wenn man eingeloggt ist.  & HTPP Status Code 200: OK \\ \cline{3-5} 
                  &                   & test-pagination-user & Getestet, ob der Benutzer die zwei Zugriffsrequests sieht, die ihm gesendet wurden. & Anzahl der gezeigten Requests auf der Seite - 2 \\ \cline{3-5} 
                  &                   & test-pagination-admin-page-1 & Getestet, ob der Admin  vier von den fünf Requests, die ihm gesendet wurden,  auf der ersten Seite sieht. & Anzahl der gezeigten Requests auf der ersten Seite - 4 \\ \cline{3-5} 
                  &                   & test-pagination-admin-page-2 & Getestet, ob der Admin  der fünfte von den fünf Requests, die ihm gesendet wurden,  auf der zweiten Seite sieht.& Anzahl der gezeigten Requests auf der zweiten Seite - 1 \\  \hline
   
                  
\multirow{3}{*}{} 5& \multirow{3}{*}{} TestMyResourcesView& test-not-logged-in & Getestet, ob die ,,My Resources``-Seite gezeigt wird, wenn man nicht eingeloggt ist.& HTTP Status Code 302: Der Benutzer wird zur ,,Login``-Seite weitergeleitet \\ \cline{3-5} &   & test-normal & Getestet, ob die ,,My Resources``-Seite gezeigt wird, wenn man eingeloggt ist.  & HTPP Status Code 200: OK \\ \cline{3-5}& & test-resources-shown & Getestet, ob der Benutzer seine zwei Ressourcen auf der ,,My Resources``-Seite  sieht. &  Anzahl der gezeigten Ressourcen auf der Seite - 2  \\ \hline


\multirow{6}{*}{}6 & \multirow{6}{*}{} TestSendDeletionRequestView& test-not-logged-in & Getestet, ob man einen Löschrequest senden kann, wenn man nicht eingeloggt ist.& HTTP Status Code 302: Der Benutzer wird zur ,,Login``-Seite weitergeleitet \\ \cline{3-5} 
                  &                   & test-not-existing-resource &  Getestet, ob man einen Löschrequest für eine Ressource senden kann, wenn die Ressource nicht existiert  & HTTP Status Code 404: Page not found  \\ \cline{3-5} 
                  &                   & test-not-owner &Getestet, ob ein Benutzer, der die Ressource nicht besitzt, einen Löschrequest senden kann. & HTTP Status Code 403: Der Benutzer wird zur ,,Permission Denied``-Seite weitergeleitet    \\ \cline{3-5} 
                  &                   & test-staff-user &Getestet, ob der Admin ein Löschrequest senden kann.  &HTTP Status Code 302: Der Admin wird zur ,,My Resources``-Seite weitergeleitet \\ \cline{3-5} 
                  &                   & test-deletion-request-exists  & Getestet, was passiert, wenn man einen Löschrequest sendet und ein Löschrequest für diese Ressource schon existiert.  & HTTP Status Code 302: Der Benutzer wird zur ,,My Resources``-Seite weitergeleitet   \\ \cline{3-5} 
                  &                  & test-normal  & Getestet, was im normalen Fall passiert, wenn man einen Löschrequest sendet. & HTTP Status Code 302: Der Benutzer wird zur ,,My Resources``-Seite weitergeleitet   \\ \hline
                  
                  
\multirow{6}{*}{}7& \multirow{6}{*}{} TestCancelDeletionRequestView& test-not-logged-in & Getestet, ob man einen Löschrequest stornieren kann, wenn man nicht eingeloggt ist.& HTTP Status Code 302: Der Benutzer wird zur ,,Login``-Seite weitergeleitet. \\ \cline{3-5} 
                  &                   & test-not-existing-resource  & Getestet, was passiert, wenn  man versucht für eine nicht existierte Ressource einen Löschrequest zu stornieren.  & HTTP Status Code 404: Page not found  \\ \cline{3-5} 
                  &                   & test-not-owner &Getestet, ob ein Benutzer, der die Ressource nicht besitzt, einen Löschrequest stornieren kann. & HTTP Status Code 403: Der Benutzer wird zur ,,Permission Denied``-Seite weitergeleitet   \\ \cline{3-5} 
                  &                   & test-staff-user &Getestet, ob der Admin einen Löschrequest stornieren kann (Admin kann keine Löschrequests senden).  &HTTP Status Code 302: Der Benutzer wird zur ,,My Resources``-Seite weitergeleitet.  \\ \cline{3-5} 
                  &                   & test-deletion-request-doesnt-exist  & Getestet was passiet, wenn man versucht einen nicht existierten Löschrequest zu stornieren.  &  HTTP Status Code 404: Page not found  \\ \cline{3-5}
                  &                   & test-normal  & Getestet, was im normalen Fall passiert, wenn man einen Löschrequest storniert. & HTTP Status Code 302: Der Benutzer wird zur ,,My Resources``-Seite weitergeleitet    \\ \hline
                  
                  
                  
\multirow{4}{*}{}8 & \multirow{4}{*}{} TestApproveAccessRequestView& test-not-logged-in & Getestet, ob man einen Zugriffsrequest genehmigen kann, wenn man nicht eingeloggt ist.& HTTP Status Code 302: Der Benutzer wird zur ,,Login``-Seite weitergeleitet   \\ \cline{3-5}  
                  &                   & test-not-existing-request  & Getestet, was passiert, wenn man versucht einen nicht existierten Zugriffsrequest zu genehmigen.  &  HTTP Status Code 404: Page not found  \\ \cline{3-5}
                  &                   & test-not-owner &Getestet, ob ein Benutzer, der die Ressource nicht besitzt, einen Zugriffsrequest genehmigen kann. & HTTP Status Code 403: Der Benutzer wird zur ,,Permission Denied``-Seite weitergeleitet  \\ \cline{3-5}
                  &                   & test-normal  & Getestet, was im normalen Fall passiert, wenn man einen Zugriffsrequest genehmigt. &   HTTP Status Code 302: Der Benutzer wird zur ,,Profile``-Seite weitergeleitet \\ \hline
                  
                  
                  
\multirow{4}{*}{} 9& \multirow{4}{*}{} TestDenyAccessRequestView& test-not-logged-in & Getestet, ob man einen Zugriffsrequest verweigern kann, wenn man nicht eingeloggt ist.& HTTP Status Code 302: Der Benutzer wird zur ,,Login``-Seite weitergeleitet  \\ \cline{3-5}  
                  &                   & test-not-existing-request   & Getestet, was passiert, wenn man versucht einen nicht existierten Zugriffsrequest abzulehnen.  &  HTTP Status Code 404: Page not found    \\ \cline{3-5}
                  &                   & test-not-owner &Getestet, ob ein Benutzer, der die Ressource nicht besitzt, einen Zugriffsrequest verweigern kann. & HTTP Status Code 403: Der Benutzer wird zur ,,Permission Denied``-Seite weitergeleitet  \\ \cline{3-5}
                  &                   & test-normal  & Getestet, was im normalen Fall passiert, wenn man einen Zugriffsrequest ablehnt. &   HTTP Status Code 302: Der Benutzer wird zur ,,Profile``-Seite weitergeleitet  \\ \hline
                  
                  
                  
\multirow{6}{*}{} 10& \multirow{6}{*}{} TestSendAccessRequestView& test-not-logged-in & Getestet, ob man einen Zugriffsrequest senden kann, wenn man nicht eingeloggt ist.& HTTP Status Code 302: Der Benutzer wird zur ,,Login``-Seite weitergeleitet \\ \cline{3-5} 
                  &                   & test-not-existing-resource &  Getestet, ob man einen Zugriffsrequest für eine Ressource senden kann, wenn die Ressource nicht existiert  &  HTTP Status Code 404: Page not found   \\ \cline{3-5}
				  &                   & test-reader &Getestet, ob ein Benutzer, der die Ressource schon zugreifen darf, einen Zugriffsrequest senden kann. & HTTP Status Code 302: Der Benutzer wird zur ,,Resources Overview``-Seite weitergeleitet  \\ \cline{3-5}
                  &                   & test-staff-user &Getestet, ob der Admin einen Zugriffsrequest senden kann.  &HTTP Status Code 302: Der Benutzer wird zur ,,Resources Overview``-Seite weitergeleitet  \\ \cline{3-5}  
                  &                   & test-access-request-exists  & Getestet, was passiert, wenn der Benutzer einen Zugriffsrequest  für eine Ressource sendet, wenn er schon einen gesendet hat. &  HTTP Status Code 302: Der Benutzer wird zur ,,Resources Overview``-Seite weitergeleitet    \\ \cline{3-5}
                  &                   &test-normal  & Getestet, was im normalen Fall passiert, wenn man einen Zugriffsrequest sendet.  & HTTP Status Code 302: Der Benutzer wird zur ,,Resources Overview``-Seite weitergeleitet    \\ \hline
                  
                  
                  
\multirow{6}{*}{} 11& \multirow{6}{*}{} TestCancelAccessRequestView& test-not-logged-in & Getestet, ob man einen Zugriffsrequest ablehnen kann, wenn man nicht eingeloggt ist.& HTTP Status Code 302: Der Benutzer wird zur ,,Login``-Seite weitergeleitet \\ \cline{3-5} 
                 &                   & test-not-existing-resource  & Getestet, was passiert, wenn man versucht für eine nicht existierte Ressource einen Zugriffsrequest zu stornieren.  & HTTP Status Code 404: Page not found  \\ \cline{3-5}
                  &                   & test-reader &Getestet, ob ein Benutzer, der die Ressource schon zugreifen darf, einen Zugriffsrequest stornieren kann. & HTTP Status Code 404: Page not found \\ \cline{3-5} 
                  &                   & test-staff-user &Getestet, ob der Admin einen Zugriffsrequest ablehnen kann (Der Admin kann keine Zugriffsrequests senden ).  &HTTP Status Code 404: Page not  found \\ \cline{3-5} 
                  &                   & test-deletion-request-doesnt-exist  & Getestet, wenn man versucht einen nicht existierten Zugriffsrequest zu stornieren.  &  HTTP Status Code 404: Page not  found   \\ \cline{3-5}

                  &                   &test-normal  & Getestet, was im normalen Fall passiert, wenn man einen Zugriffsrequest storniert.  & HTTP Status Code 302: Der Benutzer wird zur ,,Resources Overview``-Seite weitergeleitet    \\ \hline
                  
                  
                  
\multirow{4}{*}{} 12& \multirow{4}{*}{} TestDeleteResourceView& test-not-logged-in & Getestet, ob man eine Ressource löschen kann, wenn man nicht eingeloggt ist.& HTTP Status Code 302: Der Benutzer wird zur ,,Login``-Seite weitergeleitet   \\ \cline{3-5} 
                  &                   & test-not-existing-resource  & Getestet, ob man eine nicht existierte Ressource  löschen kann.  &HTTP Status Code 404: Page not found \\ \cline{3-5} 
				  &                   & test-not-staff-user &Getestet, ob der Benutzer, der kein Admin ist, eine Ressource löschen kann.  &HTTP Status Code 302: Der Benutzer wird zur ,,My Resources``-Seite weitergeleitet     \\ \cline{3-5} 
                  &                   & test-normal  & Getestet, was im normalen Fall passiert, wenn man eine Ressource löscht.  & HTTP Status Code 302: Der Benutzer wird zur ,,My Resources``-Seite weitergeleitet    \\ \hline
                  
                  
\multirow{2}{*}{} 13& \multirow{2}{*}{} TestEditNameView& test-not-logged-in & Getestet, ob man seinen Name ändern kann, wenn man nicht eingeloggt ist.& HTTP Status Code 302: Der Benutzer wird zur ,,Login``-Seite weitergeleitet.    \\ \cline{3-5} 
                  &                   & test-normal  & Getestet, was im normalen Fall passiert, wenn man seinen Name ändert.  & HTTP Status Code 302: Der Benutzer wird zur ,,Profile``-Seite weitergeleitet.    \\ \hline
                  
                  
\multirow{3}{*}{} 14& \multirow{3}{*}{} TestResourcesOverview& test-not-logged-in & Getestet, ob die ,,Resources Overview``-Seite gezeigt wird, wenn man nicht eingeloggt ist.& HTTP Status Code 302: Der Benutzer wird zur ,,Login``-Seite weitergeleite   \\ \cline{3-5} 
                  &                   &test-normal  & Getestet, ob die ,,Resources Overview``-Seite gezeigt wird, wenn man eingeloggt ist. &  HTPP Status Code 200: OK  \\ \cline{3-5} 
                  &                   &test-pagination-user  & Getestet, ob der Benutzer die vier existierten Ressourcen auf der Seite sieht. & Anzahl der gezeigten Ressourcen auf der seite-4   \\ \hline
                  
                  
\multirow{4}{*}{}15 & \multirow{4}{*}{} TestResourcesOverviewSearch& test-not-logged-in & Getestet, ob die ,,Resources Overview Search``-Seite gezeigt wird, wenn man nicht eingeloggt ist. & HTTP Status Code 302: Der Benutzer wird zur ,,Login``-Seite weitergeleite   \\ \cline{3-5} 
                  &                   &test-normal  & Getestet, ob die ,,Resources Overview Search``-Seite gezeigt wird, wenn man eingeloggt ist. &  HTPP Status Code 200: OK     \\ \cline{3-5} 
                  &                   &test-no-query  & Getestet, was passiert, wenn der Benutzer kein Zeichen für die Suchabfrage eingibt& HTTP Status Code 302: Der Benutzer wird zur ,,Resources Overview``-Seite weitergeleite   \\ \cline{3-5} 
                  &                   & test-valid-query & Getestet, ob der Benutzer alle Ressourcen, deren Namen die eingegebene Zeichen für die Suchabfrage enthalten, sieht.& Anzahl der gezeigten Ressourcen auf der Seite - 1    \\ \hline
                  
                  
\multirow{8}{*}{}16& \multirow{8}{*}{} TestPermissionEditingView& test-not-logged-in &  Getestet, ob man die ,,Edit Permissions``-Seite  einer Ressource sehen kann, wenn man nicht eingeloggt ist.& HTTP Status Code 302: Der Benutzer wird zur ,,Login``-Seite weitergeleitet   \\ \cline{3-5}  
                  &                   & test-not-existing-resource-get &  Getestet, ob man die ,,Edit Permissions``-Seite einer Ressource sehen kann, wenn die Ressource nicht existiert.  & HTTP Status Code 404: Page not found  \\ \cline{3-5} 
                  &                   & test-not-existing-resource-post &  Getestet, ob man die Rechte für eine Ressource ändern kann, wenn die Ressource nicht existiert.  &  HTTP Status Code 404: Page not found  \\ \cline{3-5} 
                  &  & test-not-authorized-user &Getestet, ob der Benutzer die ,,Edit Permissions``-Seite einer Ressource sehen kann, wenn er kein Besitzer der Ressource ist.  &  HTTP Status Code 403: Der Benutzer wird zur ,,Permission Denied``-Seite weitergeleitet  \\ \cline{3-5} 
                 &   & test-normal-get & Getestet, ob die ,,Edit Permissions``-Seite im normalen Fall gezeigt wird.  & HTPP Status Code 200: OK   \\ \cline{3-5} 
                              &   & test-normal-post & Getestet, ob  man die Rechte für eine Ressource im normalen Fall ändern kann.  & HTTP Status Code 302: Der Benutzer wird zur ,,My Resources``-Seite weitergeleitet  \\ \cline{3-5}  
                  &                   & test-pagination-users & Getestet, ob die gezeigten Benutzer (Rechte von Benutzern) auf der Seite zwei sind. & Anzahl der gezeigten Benutzer auf der Seite - 2 \\ \hline
                  
                  
\multirow{8}{*}{} 17& \multirow{8}{*}{} TestPermissionEditingViewSearch& test-not-logged-in &  Getestet, ob man die ,,Edit Permissions Search``-Seite einer Ressource sehen kann, wenn man nicht eingeloggt ist.& HTTP Status Code 302: Der Benutzer wird zur ,,Login``-Seite weitergeleitet   \\ \cline{3-5}  
                  &                   & test-not-existing-resource-get &  Getestet, ob man die ,,Edit Permissions Search``-Seite  einer Ressource sehen kann, wenn die Ressource nicht existiert.  & HTTP Status Code 404: Page not found  \\ \cline{3-5} 
                  &                   & test-not-existing-resource-post &  Getestet, ob man die Rechte für eine Ressource ändern kann, wenn die Ressource nicht existiert.  &  HTTP Status Code 302: Der Benutzer wird zur ,,Home``-Seite weitergeleitet  \\ \cline{3-5} 
                  &  & test-not-authorized-user &Getestet, ob der Benutzer die ,,Edit Permissions Search``-Seite  einer Ressource sehen kann, wenn er kein Besitzer der Ressource ist.  &  HTTP Status Code 403: Der Benutzer wird zur ,,Permission Denied``-Seite weitergeleitet  \\ \cline{3-5} 
                  &                   &test-no-query  &  Getestet, was passiert, wenn der Benutzer kein Zeichen für die Suchabfrage eingibt& HTTP Status Code 302: Der Benutzer wird zur ,,My Resources``-Seite weitergeleite  \\ \cline{3-5}
                 &   & test-normal-get & Getestet, ob die ,,Edit Permissions Search``-Seite im normalen Fall gezeigt wird.  & HTPP Status Code 200: OK   \\ \cline{3-5} 
                                &                   & test-valid-query & Getestet, ob man allen Benutzer (Rechten von Benutzern), deren Namen die eingegebenen Zeichen für die Suchabfrage enthalten, sieht.& Anzahl der gezeigten Benutzer auf der Seite - 1    
       \\ \cline{3-5}                       &   & test-normal-post & Getestet, ob  man die Rechte für eine Ressource im normalen Fall ändern kann.  & HTTP Status Code 302: Der Benutzer wird zur ,,My Resources``-Seite weitergeleitet  \\ \hline 

                  
                  
\multirow{3}{*}{} 18& \multirow{7}{*}{} TestAddNewResourceView& test-not-logged-in & Getestet, ob die ,,Add New Resource``-Seite gezeigt wird, wenn man nicht eingeloggt ist.& HTTP Status Code 302: Der Benutzer wird zur ,,Login``-Seite weitergeleitet  \\ \cline{3-5} &   & test-normal & Getestet, ob die ,,Add New Resource``-Seite gezeigt wird, wenn man eingeloggt ist.  & HTPP Status Code 200: OK \\ \hline 
                  &                   & test-no-resource-form & Getestet, was passiert, wenn man eine Post-Anfrage ohne Resourcenlink sendet.   & HTTP Status Code 302: Der Benutzer wird zur ,,My Resources``-Seite weitergeleitet   \\ \hline
                  
                  
\multirow{4}{*}{} 19& \multirow{7}{*}{} TestOpenResourceView& test-not-logged-in & Getestet, ob die ,,Open Resource``-Seite gezeigt wird, wenn man nicht eingeloggt ist. & HTTP Status Code 302: Der Benutzer wird zur ,,Login``-Seite weitergeleitet  \\ \cline{3-5}
 &                   & test-not-reader & Getestet, ob der Benutzer die ,,Open Resource``-Seite sehen kann, wenn er keine Leserechte für die Ressource hat.  & HTTP Status Code 403: Der Benutzer wird zur ,,Permission Denied``-Seite weitergeleitet   \\ \cline{3-5} 
 &   & test-normal & Getestet, ob die ,,Open Resource``-Seite im normalen Fall gezeigt wird.  & HTPP Status Code 200: OK \\ \hline
                  
                  
\multirow{4}{*}{}20 & \multirow{4}{*}{} TestApproveDeletionRequestView& test-not-logged-in & Getestet, ob man einen Löschrequest genehmigen kann, wenn man nicht eingeloggt ist.& HTTP Status Code 302: Der Benutzer wird zur ,,Login``-Seite weitergeleitet   \\ \cline{3-5}  
                  &                   & test-not-existing-request  & Getestet, was passiert, wenn man versucht einen nicht existierten Löschrequest zu genehmigen.  &  HTTP Status Code 404: Page not found   \\ \cline{3-5}
                  &                   & test-not-admin &Getestet, ob ein Benutzer, der kein Admin ist, einen Löschrequest genehmigen kann. & HTTP Status Code 403: Der Benutzer wird zur ,,Permission Denied``-Seite weitergeleitet  \\ \cline{3-5}
                  &                   & test-normal  & Getestet, was im normalen Fall passiert, wenn man einen Löschrequest genehmigt. &   HTTP Status Code 302: Der Benutzer wird zur ,,Profile``-Seite weitergeleitet \\ \hline
                  
                  
                  
\multirow{4}{*}{} 21& \multirow{4}{*}{} TestDenyDeletionRequestView& test-not-logged-in & Getestet, ob man einen Löschrequest verweigern kann, wenn man nicht eingeloggt ist.& HTTP Status Code 302: Der Benutzer wird zur ,,Login``-Seite weitergeleitet  \\ \cline{3-5}  
                  &                   & test-not-existing-request   & Getestet, was passiert, wenn man versucht einen nicht existierten Löschrequest abzulehnen.  &  HTTP Status Code 404: Page not  found   \\ \cline{3-5}
                  &                   & test-not-admin &Getestet, ob ein Benutzer, der kein Admin ist, einen Löschrequest verweigern kann. & HTTP Status Code 403: Der Benutzer wird zur ,,Permission Denied``-Seite weitergeleitet  \\ \cline{3-5}
                  &                   & test-normal  & Getestet, was im normalen Fall passiert, wenn man einen Löschrequest ablehnt. &   HTTP Status Code 302: Der Benutzer wird zur ,,Profile``-Seite weitergeleitet  \\ \hline
                  
                  
\end{longtable}
\newpage


\section{Integration-Test} \label{integrationtest}
Integration-Tests dienen dazu, das Zusammenspiel aller Komponenten zu \"uberpr\"ufen. W\"ahrend dieser Phase werden Tests entsprechend der Modularit\"at erstellt, welche im Entwurf festgelegt wurden. Django-Applications arbeiten mit dem Prinzip ,,Model View Template``, welches die Benutzeraktionen des Templates mit Hilfe der View entnimmt und diese direkt in die Datenbank integriert.\\
Um diese Verbindung zu testen, haben wir automatische Unittests benutzt (Tabelle \ref{it-tabelle} ), die den Datenbankzustand vor und nach der Ausf\"urung der View-Funktionen pr\"ufen.\\

\begin{longtable}[c]{|p{0.4cm}|P{2cm}|p{2cm}|p{5cm}|p{4.5cm}|}
\caption{Integration-Tests. Jede Klasse enthält eine Gruppe von Methoden (Testcases), die mit entsprechender Datenart arbeiten. Jede Methode testet einen Vorgang, der in der Beschreibung erläutert wird. Die Ergebnis-Spalte beschreibt das erwartete Verhalten des Systems.}
\label{it-tabelle}\\
\hline
\textbf{\#} & \textbf{Klasse}&\centerline{\textbf{Methode}}& \centerline{\textbf{Beschreibung}} & \centerline{\textbf{Ergebnis}} \\ \hline
\endfirsthead
%
\endhead
%
\multirow{3}{*}{}1 & \multirow{2}{*}{} TestUserdata & 
test-edit-name & Test öffnet die Profilseite des eingeloggten Benutzers und ändert seinen Vor- und Nachnamen. & Neue Vor- und Nachnamen werden in der Datenbank gespeichert.
\\ \cline{3-5} &   & test-my-requests & Test öffnet die Profilseite des eingeloggten Benutzers und fragt die Liste der angezeigten Requests ab. & Die angezeigte Liste entspricht der Liste aller Requests in der Datenbank, die an den Benutzer adressiert sind.
\\ \cline{3-5} &   & test-my-resources & Test öffnet die ,,My Resources``-Seite des eingeloggten Benutzers und fragt die Liste der angezeigten Ressourcen ab. & Die angezeigte Liste entspricht der Liste aller Ressourcen in der Datenbank, für die der Benutzer Besitzerrechte hat. \\ \hline

\multirow{3}{*}{} 2& \multirow{3}{*}{} TestResourcesData&  
test-resources-overview & Test öffnet die ,,Resources Overview``-Seite und fragt die Liste der angezeigten Ressourcen ab. & Die Liste enthält alle Ressourcen, die in der Datenbank gespeichert sind.  
\\ \cline{3-5} & & test-access-permissions & Test öffnet die ,,Resources Overview``-Seite und fragt die Liste der Ressourcen, für die der ,,Access``-Button angezeigt wird, ab. & Die angezeigte Liste entspricht der Liste aller Ressourcen in der Datenbank, für die der Benutzer Zugriffsrechte hat.

\\ \cline{3-5} & & test-resource-permissions & Test öffnet die ,,Edit permissions``-Seite für jede Ressource, die der eingeloggte Benutzer besitzt und fragt die Listen der angezeigten Besitzer- und Zugriffsrechte ab. & Die Listen entsprechen der Listen aller Benutzer, die als Besitzer, bzw. Leser dieser Ressource in der Datenbank gespeichert sind. \\ \hline

\multirow{8}{*}{} 3& \multirow{3}{*}{} TestRequestsData&  
test-create-access-request & Test öffnet die ,,Resources Overview``-Seite und sendet einen Zugriffsrequest für jede Ressource, für die kein ,,Access``-Button angezeigt wird. & Für alle Ressourcen, für die der Benutzer keine Zugriffsrechte hat, werden Zugriffsrequests in der Datenbank gespeichert. 
\\ \cline{3-5} & & test-cancel-access-request & Test öffnet die ,,Resources Overview``-Seite und löscht alle Zugriffsrequests, die angezeigt werden. & In der Datenbank  sind keine Zugriffsrequests von dem Benutzer gespeichert.
\\ \cline{3-5} & & test-create-deletion-request. & Test öffnet die ,,My Resources``-Seite des eingeloggten Benutzers und sendet einen Löschrequest für jede Ressourcen, die angezeigt wird. & In der Datenbank sind Löschrequests für alle Ressourcen, für die der Benutzer Besitzerrechte hat, gespeichert.  
\\ \cline{3-5} & & test-cancesl-deletion-request & Test öffnet die ,,My Resources``-Seite des eingeloggten Benutzers und löscht alle Löschrequests, die angezeigt werden. & In der Datenbank  sind keine Löschrequests von dem Benutzer gespeichert.
\\ \cline{3-5} & & test-accept-access-request & Test öffnet die Profilseite des eingeloggten Benutzers und nimmt alle angezeigten Zugriffsrequests an. & In der Datenbank sind Zugriffsrechte für entsprechende Sender und Ressourcen gespeichert. Bearbeitete Requests werden aus der Datenbank gelöscht.
\\ \cline{3-5} & & test-deny-access-request & Test öffnet die Profilseite des eingeloggten Benutzers und lehnt alle angezeigten Zugriffsrequests ab. & In der Datenbank sind keine Zugriffsrechte für entsprechende Sender und Ressourcen gespeichert. Bearbeitete Requests werden aus der Datenbank gelöscht.
\\ \cline{3-5} & & test-accept-deletion-request & Test öffnet die Profilseite des eingeloggten Admins und nimmt alle angezeigten Löschrequests an. & Bearbeitete Requests und entsprechende Ressourcen werden aus der Datenbank gelöscht.
\\ \cline{3-5} & & test-deny-deletion-request & Test öffnet die Profilseite des eingeloggten Admins und lehnt alle angezeigten Löschrequests ab. & Bearbeitete Requests werden  aus der Datenbank gelöscht und entsprechende Ressourcen bleiben in der Datenbank gespeichert.\\ \hline
                  
                  

\end{longtable}

\newpage
\section{Systemtest} \label{systemtest}
Systemtest im Sinne der Software-Entwicklung ist der abschließende Test, der vom Entwickler durchgeführt wird. Das Ziel dieses Tests ist das Verhalten des Produkts als Ganzes in einer realen Umgebung (immer noch ohne Kunden) zu beobachten.\\
Im Rahmen unseres Projekts bezeichnen wir manuelle Tests als Systemtests. Sie entsprechen möglichen Szenarien, die durch Interaktion des Benutzers mit dem System entstehen. Als Basis dafür dienen die im Pflichtenheft vordefinierte globale Testfälle. Für die klare Unterscheidung zwischen den verschiedenen Rollen, die man im Portal haben kann, sind die manuelle Tests in drei Gruppen aufgeteilt - solche die Benutzer \ref{manTestsBenutzer}, Besitzer \ref{manTestsBesitzer} oder Administrator \ref{manTestsAdmin} betreffen.\\
Um die Kompatibilität des Produkts mit verschiedenen Umgebungen zu testen, haben wir die Systemtests erfolgreich mit vier modernen Browser (deren Benutzung schon im Pflichtenheft als Kriterium gesetzt wurde) - Chrome, Firefox, Safari und Microsoft Edge auf vier Betriebssysteme - Windows 8, Windows 10, Mac und Linux durchgeführt.  
 

\begin{longtable}[c]{|P{2.5cm}|p{1.0cm}|p{5cm}|p{6.5cm}|}
\caption{Manuelle Tests für Benutzer: Jeder Testfall (Test case) testet die entsprechende funktionale Anforderung (FA) aus dem Pflichtenheft}
\label{manTestsBenutzer}\\
\hline
\textbf{Testcase}&\textbf{FA}& \textbf{Action} & \textbf{Reaction} \\ \hline
\endfirsthead
%
\endhead
%
\multirow{2}{*}{} T010 : Profilübersicht& \multirow{2}{*}{} F010 & Einloggen & ,,Profile``-Button ist angezeigt  \\ \cline{3-4}    &  & Klicke auf ,,Profile``-Button  & Profilseite ist angezeigt: User Name, Requests list, ,,My Resources``-Button \\ \cline{3-4}    &  & Ausloggen  & Profilseite kann nicht geöffnet werden \\ \hline

\multirow{2}{*}{} T020 : Name editieren & \multirow{2}{*}{} F020 & Einloggen & ,,Profile``-Button ist angezeigt  \\ \cline{3-4}    &  & Klicke auf ,,Profile``-Button  & Profilseite ist angezeigt: User Name, Requests list, ,,My Resources``-Button \\ \cline{3-4}    &  & Klicke auf den Name & Name und Vorname Felder sind angezeigt \\ \cline{3-4}    &  & Schreibe den Name und Vorname  &  Die neue Name und Vorname werden in der Felder geschrieben \\ \cline{3-4}    &  & Klicke auf ,,OK``-Button  & die neue Name und Vorname werden im Profilseite korrekt angezeigt \\ \hline

\multirow{2}{*}{} T030 : Ressourcenzugriff& \multirow{2}{*}{} F030 & Einloggen & ,,Profile``-Button ist angezeigt  \\ \cline{3-4}    &  & Klicke auf ,,Resources Overview``-Button  & Die Liste von Ressourcen wird angezeigt \\ \cline{3-4}    &  & Klicke auf ,,Access``-Button  & Herunterladenfenster wird geöffnet \\ \cline{3-4}    &  & Klicke auf ,,speichern``-Button  & Ressource wird heruntergeladen \\ \hline

\multirow{2}{*}{} T040 : Ressourcenerstellung & \multirow{2}{*}{} F040 & Einloggen & ,,Profile``Button ist angezeigt  \\ \cline{3-4}    &  & Klicke auf ,,Profile``-Button  & Profilseite ist angezeigt: User Name, Requests list, ,,My Resources``-Button \\ \cline{3-4}    &  & Klicke auf ,,My Resources``-Button  & Erstelle eine Ressource ,,Add``Button (,,+``-Button) und die Liste von Ressourcen sind angezeigt \\ \cline{3-4}    &  & Klicke auf ,,Add``-Button  & ,,Name``-, ,,Type``-, ,,Description``-Felder , ,,Link`` und ,,Add``-Button sind angezeigt \\ \cline{3-4}    &  & Fülle die Felder aus, füge ein link hinzu and klicke auf  ,,Add``-Button  & Auf  ,,My Resources``-Seite weitergeleitet, erstellte Ressource ist  in der Liste verfügbar \\ \hline

\multirow{2}{*}{} T050 : Access Request senden& \multirow{2}{*}{} F050 & Einloggen & ,,Resources Overview``-Button ist angezeigt  \\ \cline{3-4}    &  & Klicke auf ,,Resources Overview``-Button  & ,,Resources``-Seite ist angezeigt , einschließlich einer Liste von Ressourcen und einem Suchfeld \\ \cline{3-4}    &  & Klicke auf ,,Send Request``-Button  & Nachricht Dialog ist angezeigt , mit freiwilligen Beschreibungsfeld  , ,,Yes`` und  ,,Close``-Button \\ \cline{3-4}    &  & Klicke auf ,,Yes``-Button  & ,,Cancel Request``-Button ist nun angezeigt anstatt von ,,Send Request``-Button \\ \hline

\multirow{2}{*}{} T060 : Benachrichtigung& \multirow{2}{*}{} F060 & Einloggen & ,,Resources Overview``-Button ist angezeigt  \\ \cline{3-4}    &  & Klicke auf ,,Resources Overview``-Button  & ,,Resources``-Seite ist angezeigt , einschließlich einer Liste von Ressourcen und einem Suchfeld \\ \cline{3-4}    &  & Klicke auf ,,Send Request``-Button &,,Cancel Request``-Button ist nun angezeigt anstatt von ,,Send Request``-Button \\ \cline{3-4}    &  & Anfrage Akzeptieren/ablehnen (Aus dem Besitzer Konto)  & \multirow{2}{*}{} Die Logdatei enthält Email und Beschreibung  \\ \cline{3-3}    &  & Öffne die Logdatei  & \\ \cline{3-3}    &  & Suche nach dem Email  &  \\ \hline

\multirow{2}{*}{} T080 : Multiple Request& \multirow{2}{*}{} F080 & Einloggen & ,,Resources Overview``-Button ist angezeigt  \\ \cline{3-4}    &  & Klicke auf ,,Resources Overview``-Button  & ,,Resources``-Seite ist angezeigt , einschließlich einer Liste von Ressourcen und einem Suchfeld \\ \cline{3-4}    &  & Klicke auf ,,Send Request``-Button  & Nachricht Dialog ist angezeigt , mit freiwilligen Beschreibungsfeld  , ,,Yes``- und  ,,Close``-Button \\ \cline{3-4}    &  & Klicke auf ,,Yes``-Button  & ,,Cancel Request``-Button ist nun angezeigt anstatt von ,,Send Request``-Button für alle Ressourcen \\ \cline{3-4}    &  & Klicke auf ,,Send Request``-Button von einer andere Ressource  & Nachricht Dialog ist angezeigt , mit freiwilligen Beschreibungsfeld  , ,,Yes``- und ,,Close``-Button \\ \cline{3-4}    &  & Klicke auf ,,Yes``-Button  & ,,Cancel Request``-Button ist nun angezeigt anstatt von ,,Send Request``-Button für alle Ressourcen \\ \cline{3-4}    &  & Klicke auf ,,Send Request``-Button von einer anderer Ressource  & Nachricht Dialog ist angezeigt , mit freiwilligen beschreibeungsfeld  , ,,Yes``- und ,,Close``-Button \\ \cline{3-4}    &  & Klicke auf ,,Yes``-Button  & ,,Cancel Request``-Button ist nun angezeigt anstatt von ,,Send Request``-Button für alle Ressourcen \\ \cline{3-4}    &  & Klicke auf ,,Send Request``-Button von einer anderer Ressource  & Nachricht Dialog ist angezeigt , mit freiwilligen Beschreibungsfeld  , ,,Yes``- und ,,Close``-Button \\ \cline{3-4}    &  & Öffne die Logdatei  & alle Requests wurden in der Logdatei geschrieben \\ \hline
\end{longtable}


\newpage
\begin{longtable}[c]{|P{2.5cm}|p{1.0cm}|p{5cm}|p{6.5cm}|}
\caption{Manuelle Tests für Besitzer: Jeder Testfall (Test case) testet die entsprechende funktionale Anforderung (FA) aus dem Pflichtenheft}
\label{manTestsBesitzer}\\
\hline
\textbf{Testcase}&\textbf{FA}& \textbf{Action} & \textbf{Reaction} \\ \hline
\endfirsthead
%
\endhead
%
\multirow{2}{*}{} T220 : Übersicht der Requests& \multirow{2}{*}{} F180 & Einloggen & ,,Profile``-Button ist angezeigt  \\ \cline{3-4}    &  & Klicke auf ,,Profile``-Button  & Die Liste der erhaltenen Requests wird angezeigt \\ \hline

\multirow{2}{*}{} T230 : Übersicht der Ressourcen& \multirow{2}{*}{} F190 & Einloggen & ,,Profile``-Button ist angezeigt  \\ \cline{3-4}    &  & Klicke auf ,,Profile``-Button  & ,,My Resources``-Button wird angezeigt \\ \cline{3-4}    &  & Klicke auf ,,My Resources``-Button  & Liste der eigenen Ressourcen wird angezeigt  \\ \hline

\multirow{2}{*}{} T240 : Vergabe der Zugriffsrechten& \multirow{2}{*}{} F200 & Einloggen & ,,Profile``-Button ist angezeigt  \\ \cline{3-4}    &  &Auf ,,Profile``-Button klicken  & ,,My Resources``-Button wird angezeigt \\ \cline{3-4}    &  & Auf ,,My Resources``-Button klicken  &Die Liste der eigenen Ressourcen wird angezeigt, Folgende Button für jede Ressource werden angezeigt: ,,Access``, ,,Edit Permissions``, ,,Delete`` \\ \cline{3-4}    &  & Klicke auf ,,Edit Permissions``-Button  &Suchen Feld wird angezeigt \\ \cline{3-4}    &  & Schreibe den ,,Username`` und Klicke auf ,,Search``-Button  &gesuchter User wird angezeigt \\ \cline{3-4}    &  & Klicke auf ,,Reader``-Button und ,,Confirm``-Button &User soll als Leser angezeigt werden und ,,Access``-Button soll jetzt beim User Konto angezeigt werden \\ \hline

\multirow{2}{*}{} T250 : Request auf Zugriffsrechte ablehnen& \multirow{2}{*}{} F210 & Einloggen & ,,Profile``-Button ist angezeigt  \\ \cline{3-4}    &  & Klicke auf ,,Profile``-Button  & Die Liste der erhaltenen Requests wird angezeigt \\ \cline{3-4}    &  & Klicke auf ,,Deny``-Button  & Dialogbox wird angezeigt, sie enthält ein Schreibfeld, ,,Yes``- und ,,Close``-Button \\ \cline{3-4}    &  & Klicke auf ,,Yes``Button  & Request soll nicht mehr angezeigt werden und User soll nicht als Reader angezeigt werden  \\ \hline

\multirow{2}{*}{} T260 : Request auf Zugriffsrechte akzeptieren& \multirow{2}{*}{} F210 & Einloggen & ,,Profile``-Button ist angezeigt  \\ \cline{3-4}    &  & Klicke auf ,,Profile``-Button  & Die Liste der erhaltenen Requests wird angezeigt \\ \cline{3-4}    &  & Klicke auf ,,Accept``-Button  & Dialogbox wird angezeigt, sie enthält ein Schreibfeld, ,,Yes``- und ,,Close``-Button \\ \cline{3-4}    &  & Klicke auf ,,Yes``Button  & Request soll nicht mehr angezeigt werden  \\ \cline{3-4}    &  & Klicke auf ,,My Resources``-Button , ,,Edit Permission``-Button   & User soll als Leser angezeigt werden \\ \hline

\multirow{2}{*}{} T270 : Vergabe der Besitzerrechten& \multirow{2}{*}{} F220 & Einloggen & ,,Profile``-Button ist angezeigt  \\ \cline{3-4}    &  &Auf ,,Profile``-Button klicken  & ,,My Resources``-Button wird angezeigt \\ \cline{3-4}    &  & Klicke auf ,,My Resources``-Button  & Liste der eigenen Ressourcen wird angezeigt, Folgende Button für jede Ressource werden angezeigt: ,,Access``, ,,Edit Permissions``, ,,Delete`` \\ \cline{3-4}    &  & Klicke auf ,,Edit Permissions``  & Suchen Feld wird angezeigt  \\ \cline{3-4}    &  & Schreibe den ,,Username`` und Klicke auf ,,Search``-Button   & gesuchter User wird angezeigt \\ \cline{3-4}    &  & Klicke auf ,,Owner``-Button und ,,Confirm``-Button & User soll als Besitzer angezeigt werden \\ \hline

\multirow{2}{*}{} T280 : Löschrequest senden& \multirow{2}{*}{} F230 & Einloggen & ,,Profile``-Button ist angezeigt  \\ \cline{3-4}    &  &Auf ,,Profile``-Button klicken  & ,,My Resources``-Button wird angezeigt \\ \cline{3-4}    &  & Klicke auf ,,My Resources``  & Liste der eigenen Ressourcen wird angezeigt, Folgende Button für jede Ressource werden angezeigt: ,,Access``, ,,Edit Permissions``, ,,Delete`` \\ \cline{3-4}    &  & Klicke auf ,,Delete``-Button  & Dialogbox wird angezeigt , sie enthält ein Schreibfeld , ,,Yes``- und ,,Close``-Button \\ \cline{3-4}    &  & Klicke auf ,,Yes``-Button  & ,,Cancel Deletion Request``-Button wird statt ,,Delete``-Button angezeigt  \\ \hline

\multirow{2}{*}{} T290 : Löschbenachrichtigungen&  \multirow{2}{*}{} F240  & Einloggen & ,,Profile``-Button ist angezeigt  \\ \cline{3-4}    &  &Auf ,,Profile``-Button klicken  & ,,My Resources``-Button wird angezeigt \\ \cline{3-4}    &  & Klicke auf ,,My Resources``  & Liste der eigenen Ressourcen wird angezeigt, Folgende Button für jede Ressource werden angezeigt: ,,Access``, ,,Edit Permissions``, ,,Delete`` \\ \cline{3-4}    &  & Klicke auf ,,Delete``-Button  & Dialogbox wird angezeigt , sie enthält ein Schreibfeld , ,,Yes``- und ,,Close``-Button \\ \cline{3-4}    &  & Klicke auf ,,Yes``-Button  & ,,Cancel Deletion Request``-Button wird statt ,,Delete``-Button angezeigt \\ \cline{3-4}    &  & Einloggen als Administrator und klicke auf ,,Profile``-Button  & Löschrequest wird in der ,,Profile``-Seite angezeigt \\ \cline{3-4}    &  & Klicke auf ,,Accept``-Button & Dialogbox wird angezeigt , sie enthält ein Schreibfeld , ,,Yes``- und ,,Close``-Button \\ \cline{3-4}    &  & Klicke auf ,,Yes``-Button  & Löschrequest wird in der Profilseite nicht mehr angezeigt \\ \cline{3-4}    &  & Logdatei öffnen  & Logdatei enthält alle gesendete Emails, für alle Besitzer \\ \hline

\multirow{2}{*}{} T300 : Zugriffsrechte zu mehreren Nutzern vergeben& \multirow{2}{*}{} F250 & Einloggen & ,,Profile``-Button ist angezeigt  \\ \cline{3-4}    &  &Auf ,,Profile``-Button klicken  & ,,My Resources``-Button wird angezeigt \\ \cline{3-4}    &  & Klicke auf ,,My Resources``  & Liste der eigenen Ressourcen wird angezeigt, Folgende Button für jede Ressource werden angezeigt: ,,Access``, ,,Edit Permissions``, ,,Delete`` \\ \cline{3-4}    &  & Klicke auf ,,Edit Permissions``-Button  & Suchen Feld wird angezeigt \\ \cline{3-4}    &  & Schreibe "user" und Klicke auf ,,Search``-Button  & alle Users werden angezeigt \\ \cline{3-4}    &  & Klicke auf ,,Reader``-Button für alle Users und ,,Confirm``-Button  & alle Users sollen als Leser angezeigt werden  \\ \hline

\end{longtable}

\newpage
\begin{longtable}[c]{|P{2.5cm}|p{1.0cm}|p{5cm}|p{6.5cm}|}
	\caption{Manuelle Tests für Administrator: Jeder Testfall (Test case) testet die entsprechende funktionale Anforderung (FA) aus dem Pflichtenheft}
	\label{manTestsAdmin}\\
	\hline
	\textbf{Testcase}&\textbf{FA}& \textbf{Action} & \textbf{Reaction} \\ \hline
	\endfirsthead
	%
	\endhead
	%
	\multirow{2}{*}{} T110 : Löschrequest ablehnen& \multirow{2}{*}{} F110 & Einloggen & ,,Profile``-Button ist angezeigt  \\ \cline{3-4}    &  & Klicke auf ,,Profile``-Button  & Eine Liste mit Löschrequests ist zu sehen \\ \cline{3-4}    &  & ,,Deny``-Button zum gewünschten Request klicken  & Ein Dialogfenster wird angezeigt \\ \cline{3-4}    &  & Das Ablehnen bestätigen  & Die Ressource wird nicht gelöscht und Der Request wird von DB \\ \hline
	
	\multirow{2}{*}{} T120 : Löschrequest genehmigen& \multirow{2}{*}{} F110 & Einloggen & ,,Profile``-Button ist angezeigt  \\ \cline{3-4}    &  & Klicke auf ,,Profile``-Button  & Eine Liste mit Löschrequests ist zu sehen \\ \cline{3-4}    &  & ,,Accept``-Button zum gewünschten Request klicken  & Ein Dialogfenster wird angezeigt \\ \cline{3-4}    &  & Die Genehmigung  bestätigen  & Die Ressource wird von DB gelöscht und der Request wird von DB gelöscht\\ \hline
	
	\multirow{2}{*}{} T130 : Ressource löschen& \multirow{2}{*}{} F120 & Einloggen & ,,Profile``-Button ist angezeigt  \\ \cline{3-4}    &  &Auf ,,Profile``-Button klicken  & ,,My Resources``-Button ist zu sehen \\ \cline{3-4}    &  & Auf ,,My Resources``-Button klicken  &Alle Ressourcen, die der Admin besitzt, sind in einer Liste zu sehen \\ \cline{3-4}    &  & ,,Delete``-Button zur entsprechenden Ressource klicken  &Ein Dialogfenster wird angezeigt \\ \cline{3-4}    &  & Das Löschen bestätigen  &Die Ressource wird von DB gelöscht \\ \hline
	
	\multirow{2}{*}{} T140 : Besitzerrechte geben& \multirow{2}{*}{} F130 & Einloggen & ,,Manage Resources``-Button ist angezeigt  \\ \cline{3-4}    &  & ,,Manage Resources``-Button klicken  & Eine ,,Admin``-Seite wird geöffnet mit einen Link zur Liste aller Ressourcen \\ \cline{3-4}    &  & Auf ,,Resources``-Button (unter Authorizationmanagement)klicken  & Link zur Liste aller Ressourcen \\ \cline{3-4}    &  & Ressource finden und darauf klicken  & Optionen zum Editieren dieser Ressource sowie ihrer Leser und Besitzer werden angezeigt \\ \cline{3-4}    &  & Im ,,Owners``-Feld den Benutzer markieren und auf ,,Save``-Button klicken  & Der ausgewählte Benutzer ist nun Besitzer dieser Ressource \\ \hline
	
	\multirow{2}{*}{} T150 : Besitzerrechte entziehen& \multirow{2}{*}{} F130 & Einloggen & ,,Manage Resources``-Button ist angezeigt  \\ \cline{3-4}    &  & ,,Manage Resources``-Button klicken  & Eine ,,Admin``-Seite wird geöffnet mit einen Link zur Liste aller Ressourcen \\ \cline{3-4}    &  & Auf ,,Resources``-Button (unter Authorizationmanagement)klicken  & Link zur Liste aller Ressourcen \\ \cline{3-4}    &  & Ressource finden und darauf klicken  & Optionen zum Editieren dieser Ressource sowie ihrer Leser und Besitzer werden angezeigt\\ \cline{3-4}    &  & Im ,,Owners``-Feld den Benutzer demarkieren und auf ,,Save``-Button klicken  & Der ausgewählte Benutzer ist nicht mehr Besitzer dieser Ressource \\ \hline
	
	\multirow{2}{*}{} T160 : Zugriffsrechte geben& \multirow{2}{*}{} F130 & Einloggen & ,,Manage Resources``-Button ist angezeigt  \\ \cline{3-4}    &  & ,,Manage Resources``-Button klicken  & Eine ,,Admin``-Seite wird geöffnet mit einen Link zur Liste aller Ressourcen \\ \cline{3-4}    &  & Auf ,,Resources``-Button (unter Authorizationmanagement)klicken  & Link zur Liste aller Ressourcen \\ \cline{3-4}    &  & Ressource finden und darauf klicken  & Optionen zum Editieren dieser Ressource sowie ihrer Leser und Besitzer werden angezeigt \\ \cline{3-4}    &  & Im ,,Readers``-Feld den Benutzer markieren und auf ,,Save``-Button klicken  & Der ausgewählte Benutzer ist nun Leser dieser Ressource \\ \hline
	
	\multirow{2}{*}{} T170 : Zugriffsrechte entziehen& \multirow{2}{*}{} F130 & Einloggen & ,,Manage Resources``-Button ist angezeigt  \\ \cline{3-4}    &  & ,,Manage Resources``-Button klicken  & Eine ,,Admin``-Seite wird geöffnet mit einen Link zur Liste aller Ressourcen \\ \cline{3-4}    &  & Auf ,,Resources``-Button (unter Authorizationmanagement)klicken  & Link zur Liste aller Ressourcen \\ \cline{3-4}    &  & Ressource finden und darauf klicken  & Optionen zum Editieren dieser Ressource sowie ihrer Leser und Besitzer werden angezeigt \\ \cline{3-4}    &  & Im ,,Readers``-Feld den Benutzer demarkieren und auf ,,Save``-Button klicken  & Der ausgewählte Benutzer ist nicht mehr Leser dieser Ressource \\ \hline
	
	\multirow{2}{*}{} T180 : Account und Daten eines Benutzers löschen& \multirow{2}{*}{} F140 & Einloggen & ,,Manage Users``-Button ist angezeigt  \\ \cline{3-4}    &  & Auf ,,Manage Users``-Button klicken  & Eine ,,Admin``-Seite wird geöffnet mit einen Link zur Liste aller Users \\ \cline{3-4}    &  & Auf ,,Users``-Button (unter Authentication and Authorization) klicken  & Liste mit allen Benutzern im System wird angezeigt (+ Option zum Suchen) \\ \cline{3-4}    &  & Benutzer finden und auf seinen Benutzernamen klicken  & Optionen zum Editieren der Daten dieses Benutzers \\ \cline{3-4}    &  & Unten links auf ,,Delete``-Button klicken  & Eine Seite zur Bestätigung wird geöffnet \\ \cline{3-4}    &  & Löschen bestätigen  & Der ausgewählte Benutzer (und seine Requests) werden gelöscht \\ \hline
	
	\multirow{2}{*}{} T210 : Nach einem Benutzer suchen und ihn blockieren& \multirow{2}{*}{} F160, F170 & Einloggen & ,,Manage Users``-Button ist angezeigt  \\ \cline{3-4}    &  & Auf ,,Manage Users``-Button klicken  & Eine ,,Admin``-Seite wird geöffnet mit einen Link zur Liste aller Users \\ \cline{3-4}    &  & Auf ,,Users``-Button (unter Authentication and Authorization) klicken  & Liste mit allen Benutzern im System wird angezeigt (+ Option zum Suchen) \\ \cline{3-4}    &  & Benutzer finden und auf seinen Benutzernamen klicken  & Optionen zum Editieren der Daten dieses Benutzers \\ \cline{3-4}    &  & Den Haken von ,,Active`` entfernen und auf ,,Save`` Button klicken  & Der ausgewählte Benutzer kann das Portal (temporär) nicht benutzen  \\ \hline
\end{longtable}
\newpage
\section{Aufgetretene Probleme} \label{bugs}
Während der Testphase sind einige Fehler aufgetreten. Diese Fehler wurden von uns in einem Bugtracker (basiert auf Trello) erfasst und bearbeitet. Jeder Fall wurde nach der Art des Fehlers klassifiziert:

\renewcommand{\labelitemii}{$\bullet$}
\begin{itemize}
\item[] \textbf{Applikationslogik}\\\\
Folgende Fehler waren für Konsistenz des Systems kritisch und mussten umgehend behoben werden.
	\begin{itemize}
		\item \textit{Problem:}\\ Eingegebene Nachricht des Administrators beim Ablehnen eines Löschrequests wird in der Email-Benachrichtigung  nicht angezeigt.\\
			\textit{Lösung:}\\ Erweiterung des Email-Templates
		\item \textit{Problem:} \\Auf der Seite ,,Resources overview`` wird kein ,,Access``-Button für eine Ressource  angezeigt, wenn der Benutzer die Besitzerrechte für diese Ressource von einem Administrator bekommen hat.\\
		\textit{Lösung:} \\Wird der Benutzer in ,,owners``-Liste der Ressource hinzugefügt, so wird er auch zur ,,readers`` eingebunden.
		
		\item \textit{Problem:}\\
		Löschrequests von einem Benutzer werden nicht gelöscht, wenn seine Besitzerrechte entzogen werden.\\
		\textit{Lösung:}\\
		Werden Besitzerrechte entzogen, so werden auch alle existierende Löschrequests mit entsprechenden Sender und Ressource als Attribute gelöscht.
		\item \textit{Problem:}\\
		Der Zugriffsrequest von einem Benutzer wird nicht gelöscht, wenn dem die Zugriffsrechte manuell hinzugefügt werden.\\
		\textit{Lösung:}\\
		Werden Zugriffsrechte manuell hinzugefügt, so werden auch alle existierende Zugriffsrequests mit entsprechenden Sender und Ressource als Attribute gelöscht.
	\end{itemize}

\item[]\textbf{Sicherheit}\\\\
Folgende Fehler haben auf die Schwachstellen des Systems und der Datensicherheit verwiesen und mussten umgehend behoben werden.
	\begin{itemize}
		\item \textit{Problem:}\\
		Ein Benutzer kann eine Ressource zugreifen, nach dem seine Zugriffsrechte gelöscht wurden, so lange die Seite mit dem ,,Access``-Button nicht aktualisiert wird.\\
			\textit{Lösung:}\\ Zweistufige Überprüfung der Rechte.		
		\item \textit{Problem:}\\
		Einige Daten (Benutzername, Metadaten) können nach dem Ausloggen des Benutzers aus dem Browser-Cache abgerufen werden.\\
			\textit{Lösung:}\\ Cache der Seiten nicht erlauben.
	\end{itemize}

\item[] \textbf{Fehlerbenachrichtigung}\\\\
In folgenden Fällen hat angemessenes Feedback des Systems gefehlt. Diese Defekte wurden durch Weiterleitung zu entsprechenden Fehlerseiten behoben.
	\begin{itemize}
		\item Bearbeitung eines Requests, der vom Sender gelöscht wurde. 
		\item Bearbeitung eines Requests, falls die entsprechende Ressource gelöscht wurde.
	\end{itemize}
			
\item[] \textbf{Benutzerfreundlichkeit}\\\\
Folgende Defekte hatten zwar keinen Einfluss auf die Konsistenz des Systems, konnten aber einige Schwierigkeiten für Benutzer darstellen.
	\begin{itemize}
		\item \textit{Problem:}\\ Die Benutzersuche auf der ,,Edit permissions``-Seite ausschlie{\ss}lich mit dem Benutzernamen möglich.\\
			\textit{Lösung:} \\Erweiterung der Suchmethode: die Suche mit Email, Vor- und Nachnamen hinzugefügt.
		\item \textit{Problem:}\\ Ein Administrator kann Rechte für seine eigenen Ressourcen nur durch ,,Manage resources''-View ändern.\\
			\textit{Lösung:} \\Ein Administrator kann ,,Edit permissions``-View auf der Seite ,,My resources`` öffnen.
		\item \textit{Problem:}\\ Aufwendige Änderung der Rechte durch Seitenunterteilung.\\
			\textit{Lösung:} \\Im ,,Edit permissions``-View werden zehn statt vier Benutzer pro Seite angezeigt.
	\end{itemize}
	
\item[] \textbf{Benutzeroberfläche}\\\\
Folgende Defekte wurden in der Benutzeroberfläche gefunden. Behebung dieser Defekte hat die Benutzerfreundlichkeit des Produktes verbessert.
	\begin{itemize}
		\item \textit{Problem:}\\ Unbegrenzte Eingabe in einigen Textfelder möglich.\\
			\textit{Lösung:}\\ Änderung des Feldtyps und eingabe maximaler Länge.
		\item \textit{Problem:}\\ ,,Access``-Button nicht auf der ganzen Fläche anklickbar.\\
			\textit{Lösung:}\\ Löschung des Links vom Button-Überschrift.
		\item \textit{Problem:}\\ Der Name der Ressource wird auf der ,,Edit permissions``-Seite nicht angezeigt.\\
			\textit{Lösung:}\\ Erweiterung des Templates.
		\item \textit{Problem:}\\ Der Name des Benutzers kann durch Eingabe des Leerzeichen als neuen Wert gelöscht werden.\\
			\textit{Lösung:}\\ Begrenzung der erlaubten Textfeld-Mustern.
	\end{itemize}
	
\end{itemize}   

\newpage
\section{Zusammenfassung} \label{zusammenfassung}
Testüberdeckung. Fazit über die Qualität des Produkts.
\end{document}
\grid
