\documentclass[parskip=full,11pt]{scrartcl}
%\usepackage{pdfpages}
\usepackage[utf8]{inputenc}
\usepackage{amssymb}
\usepackage[T1]{fontenc}
\usepackage[german]{babel}
\usepackage[yyyymmdd]{datetime} 
\usepackage{hyperref}
\usepackage[toc, nonumberlist, automake]{} %added automake option
\usepackage{csquotes}
\usepackage{graphicx}
\usepackage{longtable}
\usepackage{multirow}
\usepackage{pbox}
\usepackage{array}
\newcolumntype{P}[1]{>{\centering\arraybackslash}p{#1}}
\hypersetup{
 		pdftitle={VForWater-Implementierung},
 }
\usepackage{fancyhdr}%<-------------to control headers and footers
\usepackage[a4paper,margin=1in,footskip=.25in]{geometry}
\fancyhf{}
\fancyfoot[C]{\thepage} %<----to get page number below text
\pagestyle{fancy} %<-------the page style itself
 
\title{Qualitätssicherung}
\subtitle{Autorisierungsmanagement für eine virtuelle Forschungsumgebung für Geodaten}
\author{Alex\\Anastasia\\Atanas\\Dannie\\ Houra\\Sonya\\}
\date{11.01.18}
 % define custom lists
\usepackage{enumitem}


\usepackage{linegoal,listings}
\newsavebox{\mylisting}
\makeatletter
\newcommand{\lstInline}[2][,]{%
	\begingroup%
	\lstset{#1}% Set any keys locally
	\begin{lrbox}{\mylisting}\lstinline!#2!\end{lrbox}% Store listing in \mylisting
	\setlength{\@tempdima}{\linegoal}% Space left on line.
	\ifdim\wd\mylisting>\@tempdima\hfill\\\fi% Insert line break
	\lstinline!#2!% Reset listing
	\endgroup%
}
\makeatother
\setlength{\parindent}{0pt}% Just for this example

\lstset{basicstyle=\footnotesize\ttfamily,breaklines=true}
\lstset{framextopmargin=50pt,frame=bottomline,showstringspaces=false,upquote=true}


 
\begin{document}
 
 \begin{titlepage}
 	
 	\begin{center}
 	\includegraphics[width=0.5\linewidth]{res/KITLogo.png}\\
 	\vspace{2cm}
 	{\scshape\LARGE\bfseries Qualitätssicherung \par}
 	\vspace{0.5cm}
 	{\scshape\Large Praxis der Softwareentwicklung\\}
 	\vspace{1cm}
 	{\scshape\Large Wintersemester 17/18\\}
 	\vspace{2cm}
 	{\huge\bfseries Autorisierungsmanagement für eine virtuelle Forschungsumgebung für Geodaten\par}
 	\vspace{2cm}
 	\vfill
 	{\bfseries {\Large Autoren}:\par}
 	{\Large Bachvarov, Aleksandar }\\
 	{\Large Dimitrov, Atanas }\\
 	{\Large Mortazavi Moshkenan, Houraalsadat }\\
 	{\Large Sakly, Khalil }\\
 	{\Large Slobodyanik, Anastasia }\\
 	{\Large Voneva, Sonya}\\
 	\vfill
 	{\large 14.03.18 \par}
 	\end{center}
 \end{titlepage}
 
 \tableofcontents
 \newpage
 \section{Einleitung}
\begin{center}
\textit{,,Testing shows the presence of bugs, not their absence``}
\end{center}
\begin{flushright}
- Edsger W. Dijkstra\\
\end{flushright}
Das Testen in der Software-Entwicklung ist der Prozess der Untersuchung des Produktes auf Unterschiede zwischen seinem beobachteten Verhalten und dem erwarteten Verhalten des modellierten Systems.
In dem Wasserfall-Vorgehensmodell wird die Testphase direkt vor der Auslieferung des Produkts durchgeführt und wird als Sicherstellung festgelegter Qualitätsanforderungen betrachtet.\\
Testphase ist mit anderen Entwicklungsphasen eng verbunden. Die erste Tests werden schon in der Implentierungsphase erstellt: Unittests zu jedem View entsprechen dem Komponenten-Test, dessen Ziel ist, Fehler in jedem Teil des Systems aufzufinden. Diese Tests werden im Kapitel \hyperref[komponententest]{\textit{Komponenten-Test}} beschrieben.\\
Das Zusammenspiel einzelnen Komponenten wird im Integration-Test geprüft. In dieser Phase wird Rückblick auf Artefakt des Entwurfs genommen, um die Integrationsstrategie während dem Test zu berücksichtigen. Tests, die diese Phase umschließt, beschreiben wir im Kapitel \hyperref[integrationtest]{\textit{Integration-Test}}.\\
Der abschließende Teil der Testphase ist System-Test, der alle Komponenten zusammen prüft. In unserem Projekt werden dafür manuelle Tests angewendet, die ermöglichen, Benutzerszenarien nah zu Realität durchzuarbeiten. Testcases (Testszenarien) in dieser Phase werden anhand funktioneller Anforderungen aus dem Pflichtenheft erstellt und im Kapitel \hyperref[systemtest]{System-Test} beschrieben.\\
In der \hyperref[zusammenfassung]{Zusammenfassung} werden Ergebnisse der Testphase betrachtet: Codeüberdeckung vor und nach der Phase, behobene Fehler und unsere Beurteilung der Lieferfertigkeit des Produktes.
 \newpage
\section{Komponenten-Test} \label{komponententest}
Um die Einzelteile unseres Projekts auf ihre Korrektheit zu überprüfen, wurden sowohl die View-Klassen als auch ihre relevante URLs durch entsprechende Unittest-Klassen getestet. Für jede Test-Klasse werden alle mögliche Fälle durch verschiedene Methoden berücksichtigt.\\
In der vorliegenden Tabelle wird beschrieben, welche Test-Klassen implementiert wurden, was in den jeweiligen Tests geprüft wird und welches Ergebnis erwartet wird.

\begin{longtable}[c]{|p{0.4cm}|P{2.5cm}|p{3.5cm}|p{5cm}|p{3cm}|}
\caption{My caption}
\label{my-label}\\
\hline
\textbf{\#} & \textbf{Testcase}&\textbf{Methoden}& \textbf{Beschreibung} & \textbf{Ergebnis} \\ \hline
\endfirsthead
%
\endhead
%
\multirow{2}{*}{}1 & \multirow{2}{*}{} TestHomeView & test-not-logged-in & Getestet, ob die Homeseite gezeigt wird, wenn man nicht eingeloggt ist. Das wird durch den zurückgegebene HTTP-Statuscode gemacht.& HTTP Status Code 302: Der Benutzer wird zur Loginseite weitergeleitet  \\ \cline{3-5} &   & test-normal & Getestet, ob die Homeseite gezeigt wird, wenn man eingeloggt ist.  & HTPP Status Code 200: OK \\ \hline

\multirow{3}{*}{} 2& \multirow{3}{*}{} TestResourceManager&  test-not-logged-in & Getestet, ob die ResourceManagerseite gezeigt wird, wenn man nicht eingeloggt ist. & HTTP Status Code 302: Der Benutzer wird zur Loginseite weitergeleitet \\ \cline{3-5} & & test-logged-in-no-admin & Getestet, ob die ResourceManagerseite gezeigt wird, wenn man eingeloggt ist, aber zwar nicht als Admin. & HTTP Status Code 302: Der Benutzer wird zur Adminloginseite weitergeleitet \\ \cline{3-5} & & test-normal & Getestet, ob die ResourceManagerseite gezeigt wird, wenn man als Admin  eingeloggt ist. & HTPP Status Code 200: OK  \\ \hline

\multirow{3}{*}{} 3& \multirow{3}{*}{} TestUserManager&  test-not-logged-in & Getestet, ob die UserManagerseite gezeigt wird, wenn man nicht eingeloggt ist. & HTTP Status Code 302: Der Benutzer wird zur Loginseite weitergeleitet  \\ \cline{3-5} & & test-logged-in-no-admin & Getestet, ob die UserManagerseite gezeigt wird, wenn man eingeloggt ist, aber zwar nicht als Admin. & HTTP Status Code 302: Der Benutzer wird zur Adminloginseite weitergeleitet \\ \cline{3-5} & & test-normal & Getestet, ob die UserManagerseite gezeigt wird, wenn man als Admin  eingeloggt ist. & HTPP Status Code 200: OK \\ \hline

\multirow{5}{*}{} 4& \multirow{5}{*}{} TestProfileView& test-not-logged-in & Getestet, ob die Profileseite gezeigt wird, wenn man nicht eingeloggt ist.& HTTP Status Code 302: Der Benutzer wird zur Loginseite weitergeleitet  \\ \cline{3-5} &   & test-normal & Getestet, ob die Profileseite gezeigt wird, wenn man eingeloggt ist.  & HTPP Status Code 200: OK \\ \cline{3-5} 
                  &                   & test-pagination-user & Getestet, ob der Benutzer die zwei Zugriffsrequesten sieht, die ihm gesendet wurden. & Anzahl der gezeigten Requeste auf der Seite - 2 \\ \cline{3-5} 
                  &                   & test-pagination-admin-page-1 & Getestet, ob der Admin  vier von den fünf Requesten, die ihm gesendet wurden,  auf der ersten Seite sieht. & Anzahl der gezeigten Requeste auf der ersten Seite - 4 \\ \cline{3-5} 
                  &                   & test-pagination-admin-page-2 & Getestet, ob der Admin  das Fünfte von den fünf Requesten, die ihm gesendet wurden,  auf der zweiten Seite sieht.& Anzahl der gezeigten Requeste auf der zweiten Seite - 1 \\  \hline
   
                  
\multirow{3}{*}{} 5& \multirow{3}{*}{} TestMyResourcesView& test-not-logged-in & Getestet, ob die MyResourcesseite gezeigt wird, wenn man nicht eingeloggt ist.& HTTP Status Code 302: Der Benutzer wird zur Loginseite weitergeleitet \\ \cline{3-5} &   & test-normal & Getestet, ob die MyResourcesseite gezeigt wird, wenn man eingeloggt ist.  & HTPP Status Code 200: OK \\ \cline{3-5}& & test-resources-shown & Getestet, ob der Benutzer seine zwei Ressourcen auf der MyResourcesseite  sieht. &  Anzahl der gezeigten Ressourcen auf der Seite - 2  \\ \hline


\multirow{6}{*}{}6 & \multirow{6}{*}{} TestSendDeletionRequestView& test-not-logged-in & Getestet, ob man ein Löschrequest senden kann, wenn man nicht eingeloggt ist.& HTTP Status Code 302: Der Benutzer wird zur Loginseite weitergeleitet \\ \cline{3-5} 
                  &                   & test-not-existing-resource &  Getestet, man ein Löschrequest für eine Ressource senden kann, wenn die Ressource nicht existiert  & HTTP Status Code 404: Page not found  \\ \cline{3-5} 
                  &                   & test-not-owner &Getestet, ob ein Benutzer, der die Ressource nicht besitzt, ein Löschrequest senden kann. & HTTP Status Code 403: Der Benutzer wird zur PermissionDeniedseite weitergeleitet    \\ \cline{3-5} 
                  &                   & test-staff-user &Getestet, ob der Admin ein Löschrequest senden kann.  &HTTP Status Code 302: Der Admin wird zur MyResourcesseite weitergeleitet \\ \cline{3-5} 
                  &                   & test-deletion-request-exists  & Getestet, was passiert, wenn ein Löschrequest für diese Ressource schon existiert.  & HTTP Status Code 302: Der Benutzer wird zur MyResourcesseite weitergeleitet   \\ \cline{3-5} 
                  &                  & test-normal  & Getestet, was es im normalen Fall passiert, wenn man eine Löschrequest sendet. & HTTP Status Code 302: Der Benutzer wird zur MyResourcesseite weitergeleitet   \\ \hline
                  
                  
\multirow{6}{*}{}7& \multirow{6}{*}{} TestCancelDeletionRequestView& test-not-logged-in & Getestet, ob man einen Löschrequest stornieren kann, wenn man nicht eingeloggt ist.& HTTP Status Code 302: Der Benutzer wird zur Loginseite weitergeleitet. \\ \cline{3-5} 
                  &                   & test-not-existing-resource  & Getestet, ob man für eine nicht existierte Ressource einen Löschrequest zu stornieren versucht.  & HTTP Status Code 404: Page not found  \\ \cline{3-5} 
                  &                   & test-not-owner &Getestet, ob ein Benutzer, der die Ressource nicht besitzt, einen Löschrequest stornieren kann. & HTTP Status Code 403: Der Benutzer wird zur PermissionDeniedseite weitergeleitet   \\ \cline{3-5} 
                  &                   & test-staff-user &Getestet, ob der Admin einen Löschrequest stornieren kann(Admin kann keine Löschrequeste senden).  &HTTP Status Code 302: Der Benutzer wird zur MyResourcesseite weitergeleitet.  \\ \cline{3-5} 
                  &                   & test-deletion-request-doesnt-exist  & Getestet was passiet, wenn man einen nicht existierten Löschrequest zu stornieren versucht.  &  HTTP Status Code 404: Page not found  \\ \cline{3-5}
                  &                   & test-normal  & Getestet, was es im normalen Fall passiert, wenn man einen Löschrequest storniert. & HTTP Status Code 302: Der Benutzer wird zur MyResourcesseite weitergeleitet    \\ \hline
                  
                  
                  
\multirow{4}{*}{}8 & \multirow{4}{*}{} TestApproveAccessRequestView& test-not-logged-in & Getestet, ob man einen Zugriffsrequest genehmigen kann, wenn man nicht eingeloggt ist.& HTTP Status Code 302: Der Benutzer wird zur Loginseite weitergeleitet   \\ \cline{3-5}  
                  &                   & test-not-existing-request  & Getestet was passiert, wenn man einen nicht existierten Zugriffsrequest zu genehmigen versucht.  &  HTTP Status Code 404: Page not found  \\ \cline{3-5}
                  &                   & test-not-owner &Getestet, ob ein Benutzer, der die Ressource nicht besitzt, einen Zugriffsrequest genehmigen kann. & HTTP Status Code 403: Der Benutzer wird zur PermissionDeniedseite weitergeleitet  \\ \cline{3-5}
                  &                   & test-normal  & Getestet, was es im normalen Fall passiert, wenn man einen Zugriffsrequest genehmigt. &   HTTP Status Code 302: Der Benutzer wird zur Profileseite weitergeleitet \\ \hline
                  
                  
                  
\multirow{4}{*}{} 9& \multirow{4}{*}{} TestDenyAccessRequestView& test-not-logged-in & Getestet, ob man einen Zugriffsrequest verweigern kann, wenn man nicht eingeloggt ist.& HTTP Status Code 302: Der Benutzer wird zur Loginseite weitergeleitet  \\ \cline{3-5}  
                  &                   & test-not-existing-request   & Getestet was passiert, wenn man einen nicht existierten Zugriffsrequest abzulehnen versucht.  &  HTTP Status Code 404: Page not found    \\ \cline{3-5}
                  &                   & test-not-owner &Getestet, ob ein Benutzer, der die Ressource nicht besitzt, einen Zugriffsrequest verweigern kann. & HTTP Status Code 403: Der Benutzer wird zur PermissionDeniedseite weitergeleitet  \\ \cline{3-5}
                  &                   & test-normal  & Getestet, was es im normalen Fall passiert, wenn man einen Zugriffsrequest ablehnt. &   HTTP Status Code 302: Der Benutzer wird zur Profileseite weitergeleitet  \\ \hline
                  
                  
                  
\multirow{6}{*}{} 10& \multirow{6}{*}{} TestSendAccessRequestView& test-not-logged-in & Getestet, ob man einen Zugriffsrequest senden kann, wenn man nicht eingeloggt ist.& HTTP Status Code 302: Der Benutzer wird zur Loginseite weitergeleitet \\ \cline{3-5} 
                  &                   & test-not-existing-resource &  Getestet, man ein Zugriffsrequest für eine Ressource senden kann, wenn die Ressource nicht existiert  &  HTTP Status Code 404: Page not found   \\ \cline{3-5}
				  &                   & test-reader &Getestet, ob ein Benutzer, der die Ressource schon zugreifen darf, einen Zugriffsrequest senden kann. & HTTP Status Code 302: Der Benutzer wird zur ResourcesOverviewseite weitergeleitet  \\ \cline{3-5}
                  &                   & test-staff-user &Getestet, ob der Admin einen Zugriffsrequest senden kann.  &HTTP Status Code 302: Der Benutzer wird zur ResourcesOverviewseite weitergeleitet  \\ \cline{3-5}  
                  &                   & test-access-request-exists  & Getestet, ob der Benutzer schon einen Zugriffsrequest gesendet hat.   &  HTTP Status Code 302: Der Benutzer wird zur ResourcesOverviewseite weitergeleitet    \\ \cline{3-5}
                  &                   &test-normal  & Getestet, was es im normalen Fall passiert, wenn man einen Zugriffsrequest sendet.  & HTTP Status Code 302: Der Benutzer wird zur ResourcesOverviewseite weitergeleitet    \\ \hline
                  
                  
                  
\multirow{6}{*}{} 11& \multirow{6}{*}{} TestCancelAccessRequestView& test-not-logged-in & Getestet, ob man einen Zugriffsrequest ablehnen kann, wenn man nicht eingeloggt ist.& HTTP Status Code 302: Der Benutzer wird zur Loginseite weitergeleitet \\ \cline{3-5} 
                 &                   & test-not-existing-resource  & Getestet, ob man für eine nicht existierte Ressource einen Zugriffsrequest zu stornieren versucht.  & HTTP Status Code 404: Page not   \\ \cline{3-5}
                  &                   & test-reader &Getestet, ob ein Benutzer, der die Ressource schon zugreifen darf, einen Zugriffsrequest stornieren kann. & HTTP Status Code 404: Page not  \\ \cline{3-5} 
                  &                   & test-staff-user &Getestet, ob der Admin einen Zugriffsrequest ablehnen kann(Der Admin kann keine Zugriffsrequeste senden ).  &HTTP Status Code 404: Page not   \\ \cline{3-5} 
                  &                   & test-deletion-request-doesnt-exist  & Getestet, ob man einen nicht existierten Zugriffsrequest zu stornieren versucht.  &  HTTP Status Code 404: Page not     \\ \cline{3-5}

                  &                   &test-normal  & Getestet, was es im normalen Fall passiert, wenn man einen Zugriffsrequest storniert.  & HTTP Status Code 302: Der Benutzer wird zur ResourcesOverviewseite weitergeleitet    \\ \hline
                  
                  
                  
\multirow{4}{*}{} 12& \multirow{4}{*}{} TestDeleteResourceView& test-not-logged-in & Getestet, ob man eine Ressource löschen kann, wenn man nicht eingeloggt ist.& HTTP Status Code 302: Der Benutzer wird zur Loginseite weitergeleitet   \\ \cline{3-5} 
                  &                   & test-not-existing-resource  & Getestet, ob man für eine nicht existierte Ressource  löschen kann.  &HTTP Status Code 404: Page not    \\ \cline{3-5} 
				  &                   & test-not-staff-user &Getestet, ob der Benutzer, der kein Admin ist, eine Ressource löschen kann.  &HTTP Status Code 302: Der Benutzer wird zur MyResourcesseite weitergeleitet     \\ \cline{3-5} 
                  &                   & test-normal  & Getestet, was es im normalen Fall passiert, wenn man eine Ressource löscht.  & HTTP Status Code 302: Der Benutzer wird zur MyResourcesseite weitergeleitet    \\ \hline
                  
                  
\multirow{2}{*}{} 13& \multirow{2}{*}{} TestEditNameView& test-not-logged-in & Getestet, ob man seinen Name ändern kann, wenn man nicht eingeloggt ist.& HTTP Status Code 302: Der Benutzer wird zur Loginseite weitergeleitet.    \\ \cline{3-5} 
                  &                   & test-normal  & Getestet, was es im normalen Fall passiert, wenn man seinen Name ändert.  & HTTP Status Code 302: Der Benutzer wird zur Profileseite weitergeleitet.    \\ \hline
                  
                  
\multirow{3}{*}{} 14& \multirow{3}{*}{} TestResourcesOverview& test-not-logged-in & Getestet, ob die ResourcesOverviewseite gezeigt ist, wen man nicht eingeloggt ist.& HTTP Status Code 302: Der Benutzer wird zur Loginseite weitergeleite   \\ \cline{3-5} 
                  &                   &test-normal  & Getestet, ob die ResourcesOverviewseite gezeigt ist, wenn man eingeloggt ist. &  HTPP Status Code 200: OK  \\ \cline{3-5} 
                  &                   &test-pagination-user  & Getestet, ob der Benutzer die vier existierten Ressourcen auf der Seite sieht. & Anzahl der gezeigten Ressourcen auf der seite-4   \\ \hline
                  
                  
\multirow{4}{*}{}15 & \multirow{4}{*}{} TestResourcesOverviewSearch& test-not-logged-in & Getestet, ob die ResourcesOverviewSearchseite gezeigt ist, wen man nicht eingeloggt ist. & HTTP Status Code 302: Der Benutzer wird zur Loginseite weitergeleite   \\ \cline{3-5} 
                  &                   &test-normal  & Getestet, ob die ResourcesOverviewSearchseite gezeigt ist, wenn man eingeloggt ist. &  HTPP Status Code 200: OK     \\ \cline{3-5} 
                  &                   &test-no-query  & Getestet was passiert, wenn der Benutzer keinen Zeichen für die Suchabfrage eingibt& HTTP Status Code 302: Der Benutzer wird zur ResourcesOverview weitergeleite   \\ \cline{3-5} 
                  &                   & test-valid-query & Getestet, ob der Benutzer die alle Ressourcen, deren Namen die eingegebene Zeichen für Suchabfrage enthalten, sieht.& Anzahl der gezeigten Ressourcen auf der Seite - 1    \\ \hline
                  
                  
\multirow{8}{*}{}16& \multirow{8}{*}{} TestPermissionEditingView& test-not-logged-in &  Getestet, ob man die EditPermissionsseite von einer Ressource sehen kann, wenn man nicht eingeloggt ist.& HTTP Status Code 302: Der Benutzer wird zur Loginseite weitergeleitet   \\ \cline{3-5}  
                  &                   & test-not-existing-resource-get &  Getestet, ob man die EditPermissionsseite von einer Ressource sehen kann, wenn die Ressource nicht existiert.  & HTTP Status Code 404: Page not found  \\ \cline{3-5} 
                  &                   & test-not-existing-resource-post &  Getestet, ob man die Rechte für eine Ressource ändern kann, wenn die Ressource nicht existiert.  &  HTTP Status Code 404: Page not found  \\ \cline{3-5} 
                  &  & test-not-authorized-user &Getestet, ob ein Benutzer die EditPermissionSeite von einer Ressource sehen kann, wenn man kein Besitzer der Ressource ist.  &  HTTP Status Code 403: Der Benutzer wird zur PermissionDeniedseite weitergeleitet  \\ \cline{3-5} 
                 &   & test-normal-get & Getestet, ob die EditPermissionseite im normalen Fall gezeigt wird.  & HTPP Status Code 200: OK   \\ \cline{3-5} 
                              &   & test-normal-post & Getestet, ob  man die Rechte für eine Ressource im normalen Fall ändern kann.  & HTTP Status Code 302: Der Benutzer wird zur MyResourcesseite weitergeleitet  \\ \cline{3-5}  
                  &                   & test-pagination-users & Getestet, ob die gezeigten Benutzer(Rechten von Benutzern) auf der Seite zwei sind. & Anzahl der gezeigten Benutzer auf der Seite - 2 \\ \hline
                  
                  
\multirow{8}{*}{} 17& \multirow{8}{*}{} TestPermissionEditingViewSearch& test-not-logged-in &  Getestet, ob man die EditPermissionsSearchseite von einer Ressource sehen kann, wenn man nicht eingeloggt ist.& HTTP Status Code 302: Der Benutzer wird zur Loginseite weitergeleitet   \\ \cline{3-5}  
                  &                   & test-not-existing-resource-get &  Getestet, ob man die EditPermissionsSearchseite von einer Ressource sehen kann, wenn die Ressource nicht existiert.  & HTTP Status Code 404: Page not found  \\ \cline{3-5} 
                  &                   & test-not-existing-resource-post &  Getestet, ob man die Rechte für eine Ressource ändern kann, wenn die Ressource nicht existiert.  &  HTTP Status Code 302: Der Benutzer wird zur Homeseite weitergeleitet  \\ \cline{3-5} 
                  &  & test-not-authorized-user &Getestet, ob ein Benutzer die EditPermissionsSearchseite von einer Ressource sehen kann, wenn man kein Besitzer der Ressource ist.  &  HTTP Status Code 403: Der Benutzer wird zur PermissionDeniedseite weitergeleitet  \\ \cline{3-5} 
                  &                   &test-no-query  &  Getestet was passiert, wenn der Benutzer keinen Zeichen für die Suchabfrage eingibt& HTTP Status Code 302: Der Benutzer wird zur MyResourcesseite weitergeleite  \\ \cline{3-5}
                 &   & test-normal-get & Getestet, ob die EditPermissionsSearchseite im normalen Fall gezeigt wird.  & HTPP Status Code 200: OK   \\ \cline{3-5} 
                                &                   & test-valid-query & Getestet, ob man allen Benutzer(Rechten von Benutzern), deren Namen die eingegebenen Zeichen für Suchabfrage enthalten, sieht.& Anzahl der gezeigten Benutzer auf der Seite - 1    
       \\ \cline{3-5}                       &   & test-normal-post & Getestet, ob  man die Rechte für eine Ressource im normalen Fall ändern kann.  & HTTP Status Code 302: Der Benutzer wird zur MyResourcesseite weitergeleitet  \\ \hline 

                  
                  
\multirow{3}{*}{} 18& \multirow{7}{*}{} TestAddNewResourceView& test-not-logged-in & Getestet, ob die AddNewResourceseite gezeigt wird, wenn man nicht eingeloggt ist.& HTTP Status Code 302: Der Benutzer wird zur Loginseite weitergeleitet  \\ \cline{3-5} &   & test-normal & Getestet, ob die AddNewResourceseite gezeigt wird, wenn man eingeloggt ist.  & HTPP Status Code 200: OK \\ \hline 
                  &                   & test-no-resource-form & Getestet was passiert, wenn man eine Post-Anfrage ohne Resourcenlink sendet.   & HTTP Status Code 302: Der Benutzer wird zur MyResourcesseite weitergeleitet   \\ \hline
                  
                  
\multirow{4}{*}{} 19& \multirow{7}{*}{} TestOpenResourceView& test-not-logged-in & Getestet, ob die OpenResourceseite gezeigt wird, wenn man nicht eingeloggt ist. Das wird durch den zurückgegebene HTTP-Statuscode gemacht.& HTTP Status Code 302: Der Benutzer wird zur ResourcesOverviewseite weitergeleitet  \\ \cline{3-5}
 &                   & test-not-reader & Getestet, ob der Benutzer die OpenResourceseite sehen kann, wenn er keine Leserechte für die Ressource hat.  & HTTP Status Code 403: Der Benutzer wird zur PermissionDeniedseite weitergeleitet   \\ \cline{3-5} 
 &   & test-normal & Getestet, ob die OpenResourceseite im normalen Fall gezeigt wird.  & HTPP Status Code 200: OK \\ \hline
                  
                  
\multirow{4}{*}{}20 & \multirow{4}{*}{} TestApproveDeletionRequestView& test-not-logged-in & Getestet, ob man einen Löschrequest genehmigen kann, wenn man nicht eingeloggt ist.& HTTP Status Code 302: Der Benutzer wird zur Loginseite weitergeleitet   \\ \cline{3-5}  
                  &                   & test-not-existing-request  & Getestet was passiert, wenn man einen nicht existierten Löschrequest zu genehmigen versucht.  &  HTTP Status Code 404: Page not    \\ \cline{3-5}
                  &                   & test-not-admin &Getestet, ob ein Benutzer, der kein Admin ist, einen Löschrequest genehmigen kann. & HTTP Status Code 403: Der Benutzer wird zur PermissionDeniedseite weitergeleitet  \\ \cline{3-5}
                  &                   & test-normal  & Getestet, was es im normalen Fall passiert, wenn man einen Löschrequest genehmigt. &   HTTP Status Code 302: Der Benutzer wird zur Profileseite weitergeleitet \\ \hline
                  
                  
                  
\multirow{4}{*}{} 21& \multirow{4}{*}{} TestDenyDeletionRequestView& test-not-logged-in & Getestet, ob man einen Löschrequest verweigern kann, wenn man nicht eingeloggt ist.& HTTP Status Code 302: Der Benutzer wird zur Loginseite weitergeleitet  \\ \cline{3-5}  
                  &                   & test-not-existing-request   & Getestet was passiert, wenn man einen nicht existierten Löschrequest abzulehnen versucht.  &  HTTP Status Code 404: Page not     \\ \cline{3-5}
                  &                   & test-not-admin &Getestet, ob ein Benutzer, der kein Admin ist, einen Löschrequest verweigern kann. & HTTP Status Code 403: Der Benutzer wird zur PermissionDeniedseite weitergeleitet  \\ \cline{3-5}
                  &                   & test-normal  & Getestet, was es im normalen Fall passiert, wenn man einen Löschrequest ablehnt. &   HTTP Status Code 302: Der Benutzer wird zur Profileseite weitergeleitet  \\ \hline
                  
                  
\end{longtable}
\newpage


\section{Integration-Test} \label{integrationtest}
Integration-Tests dienen dazu, das Zusammenspiel aller Komponenten zu \"uberpr\"ufen. W\"ahrend dieser Phase werden Tests entsprechend der Modularit\"at erstellt, welche im Entwurf festgelegt wurden. Django-Applications arbeiten mit dem Prinzip ,,Model View Template``, welches die Benutzeraktionen des Templates mit Hilfe der View entnimmt und diese direkt in die Datenbank integriert.\\
Um diese Verbindung zu testen, haben wir automatische Unittests benutzt (Tabelle \ref{it-tabelle} ), die den Datenbankzustand vor und nach der Ausf\"urung der View-Funktionen pr\"ufen.\\

\begin{longtable}[c]{|p{0.4cm}|P{2cm}|p{3cm}|p{4.cm}|p{4.5cm}|}
\caption{Integration-Tests. Jede Klasse enthält eine Gruppe von Methoden (Testcases) entsprächen der Datenart, mit denen in diesen Methoden gearbeitet wird. Jede Methode testet einen Vorgang, der in der Beschreibung erläutert wird. Ergebnis-Spalte beschreibt das erwartete Verhalten des Systems.}
\label{it-tabelle}\\
\hline
\textbf{\#} & \textbf{Klasse}&\centerline{\textbf{Methode}}& \centerline{\textbf{Beschreibung}} & \centerline{\textbf{Ergebnis}} \\ \hline
\endfirsthead
%
\endhead
%
\multirow{3}{*}{}1 & \multirow{2}{*}{} TestUserdata & 
test-edit-name & Test öffnet Profilseite des eingeloggten Benutzers und ändert seine Vor- und Nachnamen. & Neue Vor- und Nachname werden in der Datenbank gespeichert.
\\ \cline{3-5} &   & test-my-requests & Test öffnet Profilseite des eingeloggten Benutzers und fragt die Liste der angezeigten Requests ab. & Die angezeigte Liste entspricht der Liste aller Requests in der Datenbank, die an den Benutzer adressiert sind.
\\ \cline{3-5} &   & test-my-resources & Test öffnet ,,My Resources``-Seite des eingeloggten Benutzers und fragt die Liste der angezeigten Ressourcen ab. & Die angezeigte Liste entspricht der Liste aller Ressourcen in der Datenbank, für die der Benutzer Besitzerrechte hat. \\ \hline

\multirow{3}{*}{} 2& \multirow{3}{*}{} TestResourcesData&  
test-resources-overview & Test öffnet ,,Resources Overview``-Seite und fragt die Liste der angezeigten Ressourcen ab. & Die Liste enthält alle Ressourcen, die in der Datenbank gespeichert sind.  
\\ \cline{3-5} & & test-access-permissions & Test öffnet ,,Resources Overview``-Seite und fragt die Liste der Ressourcen, für die der Access-Button angezeigt wird, ab. & Die angezeigte Liste entspricht der Liste aller Ressourcen in der Datenbank, für die der Benutzer Leserechte hat.

\\ \cline{3-5} & & test-resource-permissions & Test öffnet ,,Edit permissions``-Seite für jede Ressource, die der eingeloggte Benutzer besitzt und fragt die Listen der angezeigten Besitzer- und Leserechte ab. & Die Listen entsprechen der Listen aller Benutzer, die als Besitzer, bzw. Leser dieser Ressource in der Datenbank gespeichert sind. \\ \hline

\multirow{8}{*}{} 2& \multirow{3}{*}{} TestRequestsData&  
test-create-access-request & Test öffnet ,,Resources Overview``-Seite und sendet Zugriffsrequests für alle Ressourcen, für die kein ,,Access``-Button angezeigt wird. & Für alle Ressourcen, für die der Benutzer keine Leserechte hat, werden Zugriffsrequests in der Datenbank gespeichert. 
\\ \cline{3-5} & & test-cancel-access-request & Test öffnet ,,Resources Overview``-Seite und löscht alle Zugriffsrequests, die angezeigt werden. & In der Datenbank  sind keine Zugriffsrequests von dem Benutzer gespeichert.
\\ \cline{3-5} & & test-create-deletion-request. & Test öffnet ,,My Resources``-Seite des eingeloggten Benutzers und sendet Löschrequests für alle Ressourcen, die angezeigt werden. & In der Datenbank sind Löschrequests für alle Ressourcen, für die der Benutzer Besitzerrechte hat, gespeichert.  
\\ \cline{3-5} & & test-cancesl-deletion-request & Test öffnet ,,My Resources``-Seite des eingeloggten Benutzers und löscht alle Löschrequests, die angezeigt werden. & In der Datenbank  sind keine Löschrequests von dem Benutzer gespeichert.
\\ \cline{3-5} & & test-accept-access-request & Test öffnet Profilseite des eingeloggten Benutzers und nimmt alle angezeigten Zugriffsrequests an. & In der Datenbank sind Leserechte für entsprechende Sender und Ressourcen gespeichert. Bearbeitete Requests werden aus der Datenbank gelöscht.
\\ \cline{3-5} & & test-deny-access-request & Test öffnet Profilseite des eingeloggten Benutzers und lehnt alle angezeigten Zugriffsrequests ab. & In der Datenbank sind keine Leserechte für entsprechende Sender und Ressourcen gespeichert. Bearbeitete Requests werden aus der Datenbank gelöscht.
\\ \cline{3-5} & & test-accept-deletion-request & Test öffnet Profilseite des eingeloggten Admins und nimmt alle angezeigten Löschrequests an. & Bearbeitete Requests und entsprechende Ressourcen werden aus der Datenbank gelöscht.
\\ \cline{3-5} & & test-deny-deletion-request & Test öffnet Profilseite des eingeloggten Admins und lehnt alle angezeigten Löschrequests ab. & Bearbeitete Requests werden  aus der Datenbank gelöscht und entsprechende Ressourcen bleiben in der Datenbank gespeichert.\\ \hline
                  
                  

\end{longtable}

\newpage
\section{System-Test} \label{systemtest}
System-Test im Sinne der Software-Entwicklung ist der abschließende Test, der vom Entwickler durchgeführt wird. Das Ziel dieses Tests ist das Verhalten des Produkts als Ganzes in einer realen Umgebung (immer noch ohne Kunden) zu beobachten.\\
Im Rahmen unseres Projekts sind die System-Tests als manuelle Tests bezeichnet. Sie entsprechen möglichen Szenarien, die durch Interaktion mit dem System entstehen. Für die klare Unterscheidung zwischen den verschiedenen Rollen, die man im Portal haben kann, sind die manuelle Tests in drei Gruppen aufgeteilt - für Benutzer, für Besitzer und für Administrator.

\begin{longtable}[c]{|P{2.5cm}|p{1.0cm}|p{5cm}|p{6.5cm}|}
\caption{Manuelle Tests für User}
\label{my-label}\\
\hline
\textbf{Testcase}&\textbf{FA}& \textbf{Action} & \textbf{Reaction} \\ \hline
\endfirsthead
%
\endhead
%
\multirow{2}{*}{} T010 : Profilübersicht& \multirow{2}{*}{} F010 & Einloggen & Profile Button ist angezeigt  \\ \cline{3-4}    &  & Klicke auf Profile  & Profile seite ist angezeigt: user name, requests list, my resources button \\ \cline{3-4}    &  & Ausloggen  & Profile seite kann nicht geöffnet werden \\ \hline

\multirow{2}{*}{} T020 : Name editieren& \multirow{2}{*}{} F020 & Einloggen & Profile Button ist angezeigt  \\ \cline{3-4}    &  & Klicke auf Profile  & Profile seite ist angezeigt: user name, requests list, my resources button \\ \cline{3-4}    &  & Klicke auf den Name  & Name und vorname felder sind angezeigt \\ \cline{3-4}    &  & Schreibe Name und Vorname  & Neue Name und Vorname werden in der Felder geschrieben \\ \cline{3-4}    &  & Klicke auf ok Button  & Neue Name und Vorname werden im Profile seite correct angezeigt \\ \hline

\multirow{2}{*}{} T030 : Ressourcenzugriff& \multirow{2}{*}{} F030 & Einloggen & Profile Button ist angezeigt  \\ \cline{3-4}    &  & Klicke auf Profile  & Profile seite ist angezeigt: user name, requests list, my resources button \\ \cline{3-4}    &  & Klicke auf My Resources  & Erstelle Resource Button ("+" button) und die Liste von Resourcen sind angezeigt \\ \hline

\multirow{2}{*}{} T040 : Ressourcenerstellung& \multirow{2}{*}{} F040 & Einloggen & Profile Button ist angezeigt  \\ \cline{3-4}    &  & Klicke auf Profile  & Profile seite ist angezeigt: user name, requests list, my resources button \\ \cline{3-4}    &  & Klicke auf My Resources  & Erstelle Resource Button("+" button und die Liste von Resourcen sind angezeigt \\ \cline{3-4}    &  & Klicke auf Add Resource Button ("+" button)  & Name, Type, Description felder , Link und add Button sind angezeigt \\ \cline{3-4}    &  & Fülle die Name , Type ,Description felds aus, füge ein link hinzu and klicke auf  Add Button  & Auf  my Resources-Seite weitergeleitet,erstellte Resource ist  in der Liste verfügbar \\ \hline

\multirow{2}{*}{} T050 : Access Request senden& \multirow{2}{*}{} F050 & Einloggen & Resources Overview Button ist angezeigt  \\ \cline{3-4}    &  & Klicke auf Resources Overview  & Resources-seite ist angezeigt , einschließlich einer Liste von Ressourcen und einem Suchfeld \\ \cline{3-4}    &  & Klicke auf send Request  & Nachricht Dialog ist angezeigt , mit freiwilligen beschreibungsfeld  , Yes und  close Button \\ \cline{3-4}    &  & Klicke auf yes Button  & Cancel request Button ist nun angezeigt anstatt von send Request \\ \hline

\multirow{2}{*}{} T060 : Benachrichtigung& \multirow{2}{*}{} F060 & Einloggen & Resources Overview Button ist angezeigt  \\ \cline{3-4}    &  & Klicke auf Resources Overview  & Resources-seite ist angezeigt , einschließlich einer Liste von Ressourcen und einem Suchfeld \\ \cline{3-4}    &  & Klicke auf send Request &Cancel request Button ist nun angezeigt anstatt von send Request \\ \cline{3-4}    &  & Anfrage Acceptieren/ablehnen (Aus dem Besitzer konto)  & \multirow{2}{*}{} Logdatei enthält Email und Beschreibung  \\ \cline{3-3}    &  & Öffene das Logdatei  & \\ \cline{3-3}    &  & Suche nach dem email  &  \\ \hline

\multirow{2}{*}{} T070 : Multiple Request& \multirow{2}{*}{} F080 & Einloggen & Resources Overview Button ist angezeigt  \\ \cline{3-4}    &  & Klicke auf Resources Overview  & Resources-seite ist angezeigt , einschließlich einer Liste von Ressourcen und einem Suchfeld \\ \cline{3-4}    &  & Klicke auf send Request  & Nachricht Dialog ist angezeigt , mit freiwilligen beschreibeungsfeld  , Yes und  close Button \\ \cline{3-4}    &  & Klicke auf yes Button  & Cancel request Button ist nun angezeigt anstatt von send Request für alle Resourcen \\ \cline{3-4}    &  & Klicke auf send Request von einer andere Resource  & Nachricht Dialog ist angezeigt , mit freiwilligen beschreibeungsfeld  , Yes und  close Button \\ \cline{3-4}    &  & Klicke auf yes Button  & Cancel request Button ist nun angezeigt anstatt von send Request für alle Resourcen \\ \cline{3-4}    &  & Klicke auf send Request von einer anderer Resource  & Nachricht Dialog ist angezeigt , mit freiwilligen beschreibeungsfeld  , Yes und  close Button \\ \cline{3-4}    &  & Klicke auf yes Button  & Cancel request Button ist nun angezeigt anstatt von send Request für alle Resourcen \\ \cline{3-4}    &  & Klicke auf send Request von einer anderer Resource  & Nachricht Dialog ist angezeigt , mit freiwilligen beschreibeungsfeld  , Yes und  close Button \\ \cline{3-4}    &  & Öffene das Logdatei  & alle requests wurden in der Logdatei geschrieben \\ \hline
\end{longtable}





\newpage
\begin{longtable}[c]{|P{2.5cm}|p{1.0cm}|p{5cm}|p{6.5cm}|}
\caption{Manuelle Tests für Admin}
\label{my-label}\\
\hline
\textbf{Testcase}&\textbf{FA}& \textbf{Action} & \textbf{Reaction} \\ \hline
\endfirsthead
%
\endhead
%
\multirow{2}{*}{} T80 : Löschrequest ablehnen& \multirow{2}{*}{} F110 & Einloggen & Profile Button ist angezeigt  \\ \cline{3-4}    &  & Klicke auf Profile  & Eine Liste mit Löschrequesten ist zu sehen \\ \cline{3-4}    &  & Deny Button zum gewünschten Request klicken  & Ein Dialogfenster wird angezeigt \\ \cline{3-4}    &  & Das Ablehnen bestätigen  & Die Ressource wird nicht gelöscht
Requst wird von DB gelöscht \\ \hline

\multirow{2}{*}{} T90 : Löschrequest genehmigen& \multirow{2}{*}{} F110 & Einloggen & Profile Button ist angezeigt  \\ \cline{3-4}    &  & Klicke auf Profile  & Eine Liste mit Löschrequesten ist zu sehen \\ \cline{3-4}    &  & Deny Button zum gewünschten Request klicken  & Ein Dialogfenster wird angezeigt \\ \cline{3-4}    &  & Die Genehmigung  bestätigen  & Die Ressource wird von DB gelöscht und der Request wird von DB gelöscht
Requst wird von DB gelöscht \\ \hline

\multirow{2}{*}{} T100 : Ressource löschen& \multirow{2}{*}{} F120 & Einloggen & Button Profile ist angezeigt  \\ \cline{3-4}    &  &Auf Profile klicken  & Ein Button My Resources ist zu sehen \\ \cline{3-4}    &  & Auf My Resources klicken  &Alle Ressourcen, die der Admin besitzt, sind in einer Liste zu sehen \\ \cline{3-4}    &  & Delete Button zur ensprechenden Ressource klicken  &Ein Dialogfenster wird angezeigt \\ \cline{3-4}    &  & Das Löschen bestätigen  &Die Ressource wird von DB gelöscht \\ \hline

\multirow{2}{*}{} T110 : Besitzerrechte geben& \multirow{2}{*}{} F130 & Einloggen & Button Manage Resources ist angezeigt  \\ \cline{3-4}    &  & Auf Manage Resources klicken  & Eine Admin-Seite wird geöffnet mit einer
Link zur Liste aller Ressourcen \\ \cline{3-4}    &  & Auf Resources (unter
Authorizationmanagement)klicken  & Link zur Liste aller Ressourcen \\ \cline{3-4}    &  & Ressource finden und darauf klicken  & Optionen zum Editieren dieser Ressource sowie ihrer Leser und Besitzer \\ \cline{3-4}    &  & Im Feld Owners den Benutzer markieren und auf Save Button klicken  & Der ausgewählte Benutzer ist schon Besitzer dieser Ressource \\ \hline

\multirow{2}{*}{} T120 : Besitzerrechte entziehen& \multirow{2}{*}{} F130 & Einloggen & Button Manage Resources ist angezeigt  \\ \cline{3-4}    &  & Auf Manage Resources klicken  & Eine Admin-Seite wird geöffnet mit einer
Link zur Liste aller Ressourcen \\ \cline{3-4}    &  & Auf Resources (unter
Authorizationmanagement)klicken  & Link zur Liste aller Ressourcen \\ \cline{3-4}    &  & Ressource finden und darauf klicken  & Optionen zum Editieren dieser Ressource sowie ihrer Leser und Besitzer \\ \cline{3-4}    &  & Im Feld Owners den Benutzer demarkieren und auf Save Button klicken  & Der ausgewählte Benutzer ist nicht mehr Besitzer dieser Ressource \\ \hline

\multirow{2}{*}{} T130 : Zugriffsrechte geben& \multirow{2}{*}{} F130 & Einloggen & Button Manage Resources ist angezeigt  \\ \cline{3-4}    &  & Auf Manage Resources klicken  & Eine Admin-Seite wird geöffnet mit einer
Link zur Liste aller Ressourcen \\ \cline{3-4}    &  & Auf Resources (unter
Authorizationmanagement)klicken  & Link zur Liste aller Ressourcen \\ \cline{3-4}    &  & Ressource finden und darauf klicken  & Optionen zum Editieren dieser Ressource sowie ihrer Leser und Besitzer \\ \cline{3-4}    &  & Im Feld readers den Benutzer markieren und auf Save Button klicken  & Der ausgewählte Benutzer ist schon leser dieser Ressource \\ \hline

\multirow{2}{*}{} T140 : Besitzerrechte entziehen& \multirow{2}{*}{} F130 & Einloggen & Button Manage Resources ist angezeigt  \\ \cline{3-4}    &  & Auf Manage Resources klicken  & Eine Admin-Seite wird geöffnet mit einer
Link zur Liste aller Ressourcen \\ \cline{3-4}    &  & Auf Resources (unter
Authorizationmanagement)klicken  & Link zur Liste aller Ressourcen \\ \cline{3-4}    &  & Ressource finden und darauf klicken  & Optionen zum Editieren dieser Ressource sowie ihrer Leser und Besitzer \\ \cline{3-4}    &  & Im Feld Readers den Benutzer demarkieren und auf Save Button klicken  & Der ausgewählte Benutzer ist nicht mehr leser dieser Ressource \\ \hline

\multirow{2}{*}{} T150 : Account und Daten eines Benutzers löschen& \multirow{2}{*}{} F140 & Einloggen & Button Manage Resources ist angezeigt  \\ \cline{3-4}    &  & Auf Manage Users klicken  & Eine Admin-Seite wird geöffnet mit einer
Link zur Liste aller Users \\ \cline{3-4}    &  & Auf Users (unter
Authentication and Authorization) klicken  & Liste mit allen Benutzern im System wird angezeigt (+ Option zum Suchen) \\ \cline{3-4}    &  & Benutzer finden und auf seinen Benutzernamen klicken  & Optionen zum Editieren der Daten dieses Benutzers \\ \cline{3-4}    &  & Unten links auf Delete klicken  & Eine Seite zur Bestätigung wird geöffnet \\ \cline{3-4}    &  & Löschen bestätigen  & Der ausgewählte Benutzer (und seine Requeste) werden gelöscht \\ \hline

\multirow{2}{*}{} T160 : Nach einem Benutzer suchen und ihn blockieren& \multirow{2}{*}{} F160,
F170 & Einloggen & Button Manage Users ist angezeigt  \\ \cline{3-4}    &  & Auf Manage Users klicken  & Eine Admin-Seite wird geöffnet mit einer Link zur Liste aller Users \\ \cline{3-4}    &  & Auf Users (unter
Authentication and Authorization) klicken  & Liste mit allen Benutzern im System wird angezeigt (+ Option zum Suchen) \\ \cline{3-4}    &  & Benutzer finden und auf seinen Benutzernamen klicken  & Optionen zum Editieren der Daten dieses Benutzers \\ \cline{3-4}    &  & Den Haken von Active entfernen und auf Save Button klicken  & Der ausgewählte Benutzer kann das Portal (temporär) nicht benutzen  \\ \hline
\end{longtable}

\newpage
\begin{longtable}[c]{|P{2.5cm}|p{1.0cm}|p{5cm}|p{6.5cm}|}
\caption{Manuelle Tests für Owner}
\label{my-label}\\
\hline
\textbf{Testcase}&\textbf{FA}& \textbf{Action} & \textbf{Reaction} \\ \hline
\endfirsthead
%
\endhead
%
\multirow{2}{*}{} T170 : Übersicht der Requests& \multirow{2}{*}{} F180 & Einloggen & Profile Button ist angezeigt  \\ \cline{3-4}    &  & Klicke auf Profile  & Liste der erhaltenen Requests wird angezeigt \\ \hline

\multirow{2}{*}{} T180 : Übersicht der Ressourcen& \multirow{2}{*}{} F190 & Einloggen & Profile Button ist angezeigt  \\ \cline{3-4}    &  & Klicke auf Profile  & My Resources Button wird angezeigt \\ \cline{3-4}    &  & Klicke auf My Resources  & Liste der eigenen Ressourcen wird angezeigt  \\ \hline

\multirow{2}{*}{} T190 : Vergabe der Zugriffsrechten& \multirow{2}{*}{} F200 & Einloggen & Button Profile ist angezeigt  \\ \cline{3-4}    &  &Auf Profile klicken  & My Resources Button wird angezeigt \\ \cline{3-4}    &  & Auf My Resources klicken  &Liste der eigenen Ressourcen wird angezeigt, Folgende Button für jede Resource werden angezeigt: Access, Edit Permission, Delete \\ \cline{3-4}    &  & Klicke auf edit Permissions  &Suchen Feld wird angezeigt \\ \cline{3-4}    &  & Schreibe den Username und Klicke auf Search Button  &gesuchter User wird angezeigt \\ \cline{3-4}    &  & Klicke auf Reader Button und Confirm  &User soll als Reader angezeigt werden und Access Button soll jetzt beim User konto angezeigt werden \\ \hline

\multirow{2}{*}{} T200 : Request auf Zugriffsrechte ablehnen& \multirow{2}{*}{} F210 & Einloggen & Profil Button wird angezeigt  \\ \cline{3-4}    &  & Klicke auf Profile  & Liste der erhaltenen Requests wird angezeigt \\ \cline{3-4}    &  & Klicke auf Deny Button  & Dialogbox wird angezeiget, sie enthält ein schreibfeld, yes und close Buttons \\ \cline{3-4}    &  & Klicke auf yes Button  & Request soll nicht mehr angezeigt werden und User soll nicht als Reader angezeigt werden  \\ \hline

\multirow{2}{*}{} T210 : Request auf Zugriffsrechte acceptieren& \multirow{2}{*}{} F210 & Einloggen & Profil Button wird angezeigt  \\ \cline{3-4}    &  & Klicke auf Profile  & Liste der erhaltenen Requests wird angezeigt \\ \cline{3-4}    &  & Klicke auf Deny Button  & Dialogbox wird angezeiget, sie enthält ein schreibfeld, yes und close Buttons \\ \cline{3-4}    &  & Klicke auf yes Button  & Request soll nicht mehr angezeigt werden  \\ \cline{3-4}    &  & Klicke auf My Resources , edit Permission   & User soll als Reader angezeigt werden \\ \hline

\multirow{2}{*}{} T220 : Vergabe der Besitzerrechte& \multirow{2}{*}{} F220 & Einloggen & Profil Button wird angezeigt  \\ \cline{3-4}    &  & Klicke auf Profile  & My Resources Button wird angezeigt \\ \cline{3-4}    &  & Klicke auf My Resources  & Liste der eigenen Ressourcen wird angezeigt, Folgende Button für jede Resource werden angezeigt: Access, Edit Permission, Delete \\ \cline{3-4}    &  & Klicke auf edit Permissions  & Suchen Feld wird angezeigt  \\ \cline{3-4}    &  & Schreibe den Username und Klicke auf Search Button   & gesuchter User wird angezeigt \\ \cline{3-4}    &  & SKlicke auf owner Button und Confirm
   & User soll als ownerer angezeigt werden \\ \hline

\multirow{2}{*}{} T230 : Löschrequest senden& \multirow{2}{*}{} F230 & Einloggen & Profil Button wird angezeigt  \\ \cline{3-4}    &  & Klicke auf Profile  & My Resources Button wird angezeigt \\ \cline{3-4}    &  & Klicke auf My Resources  & Liste der eigenen Ressourcen wird angezeigt, Folgende Button für jede Resource werden angezeigt: Access, Edit Permission, Delete \\ \cline{3-4}    &  & Klicke auf Delete Button  & Dialogbox wird angezeiget , sie enthält ein schreibfeld , yes und close Buttons \\ \cline{3-4}    &  & Klicke auf yes Button  & Cancel Deletion request Button wird statt Delete button angezeigt  \\ \hline

\multirow{2}{*}{} T240 : Löschbenachrichtigungen&  \multirow{2}{*}{} F240  & Einloggen & Profil Button wird angezeigt  \\ \cline{3-4}    &  & Klicke auf Profile  & My Resources Button wird angezeigt \\ \cline{3-4}    &  & Klicke auf My Resources  & Liste der eigenen Ressourcen wird angezeigt, Folgende Button für jede Resource werden angezeigt: Access, Edit Permission, Delete \\ \cline{3-4}    &  & Klicke auf Delete Button  & Dialogbox wird angezeiget , sie enthält ein schreibfeld , yes und close Buttons \\ \cline{3-4}    &  & Klicke auf yes Button  & Cancel Deletion request Button wird statt Delete button angezeigt \\ \cline{3-4}    &  & Einloggen mit Admin und klicke auf Profil  & Deletion Request wird in der Profil-seite angezeigt \\ \cline{3-4}    &  & Klicke auf Accept & Dialogbox wird angezeiget , sie enthält ein schreibfeld , yes und close Buttons \\ \cline{3-4}    &  & Klicke auf yes  & Deletion Request wird in der Profil-seite nicht mehr angezeigt \\ \cline{3-4}    &  & Logdatei öffnen  & Logdatei enthält alle gesendete email, für alle besitzer \\ \hline

\multirow{2}{*}{} T250 : Zugriffsrechte zu mehreren Nutzern vergeben& \multirow{2}{*}{} F250 & Einloggen & Profil Button wird angezeigt  \\ \cline{3-4}    &  & Klicke auf Profile  & My Resources Button wird angezeigt \\ \cline{3-4}    &  & Klicke auf My Resources  & Liste der eigenen Ressourcen wird angezeigt, Folgende Button für jede Resource werden angezeigt: Access, Edit Permission, Delete \\ \cline{3-4}    &  & Klicke auf edit Permissions  & Suchen Feld wird angezeigt \\ \cline{3-4}    &  & Schreibe "user" und Klicke auf Search Button  & alle Users werden angezeigt \\ \cline{3-4}    &  & Klicke auf Reader Button für alle Users und Confirm  & alle Users sollen als Reader angezeigt werden  \\ \hline



\end{longtable}

\newpage
\section{Zusammenfassung} \label{zusammenfassung}
Testüberdeckung. Fazit über die Qualität des Produkts.
\end{document}
\grid
