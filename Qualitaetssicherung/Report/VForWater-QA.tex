\documentclass[parskip=full,11pt]{scrartcl}
%\usepackage{pdfpages}
\usepackage[utf8]{inputenc}
\usepackage{amssymb}
\usepackage[T1]{fontenc}
\usepackage[german]{babel}
\usepackage[yyyymmdd]{datetime} 
\usepackage{hyperref}
\usepackage[toc, nonumberlist, automake]{} %added automake option
\usepackage{csquotes}
\usepackage{graphicx}
\usepackage{longtable}
\hypersetup{
 		pdftitle={VForWater-Implementierung},
 }
\usepackage{fancyhdr}%<-------------to control headers and footers
\usepackage[a4paper,margin=1in,footskip=.25in]{geometry}
\fancyhf{}
\fancyfoot[C]{\thepage} %<----to get page number below text
\pagestyle{fancy} %<-------the page style itself
 
\title{Qualitätssicherung}
\subtitle{Autorisierungsmanagement für eine virtuelle Forschungsumgebung für Geodaten}
\author{Alex\\Anastasia\\Atanas\\Dannie\\ Houra\\Sonya\\}
\date{11.01.18}
 % define custom lists
\usepackage{enumitem}


\usepackage{linegoal,listings}
\newsavebox{\mylisting}
\makeatletter
\newcommand{\lstInline}[2][,]{%
	\begingroup%
	\lstset{#1}% Set any keys locally
	\begin{lrbox}{\mylisting}\lstinline!#2!\end{lrbox}% Store listing in \mylisting
	\setlength{\@tempdima}{\linegoal}% Space left on line.
	\ifdim\wd\mylisting>\@tempdima\hfill\\\fi% Insert line break
	\lstinline!#2!% Reset listing
	\endgroup%
}
\makeatother
\setlength{\parindent}{0pt}% Just for this example

\lstset{basicstyle=\footnotesize\ttfamily,breaklines=true}
\lstset{framextopmargin=50pt,frame=bottomline,showstringspaces=false,upquote=true}


 
\begin{document}
 
 \begin{titlepage}
 	
 	\begin{center}
 	\includegraphics[width=0.5\linewidth]{res/KITLogo.png}\\
 	\vspace{2cm}
 	{\scshape\LARGE\bfseries Qualitätssicherung \par}
 	\vspace{0.5cm}
 	{\scshape\Large Praxis der Softwareentwicklung\\}
 	\vspace{1cm}
 	{\scshape\Large Wintersemester 17/18\\}
 	\vspace{2cm}
 	{\huge\bfseries Autorisierungsmanagement für eine virtuelle Forschungsumgebung für Geodaten\par}
 	\vspace{2cm}
 	\vfill
 	{\bfseries {\Large Autoren}:\par}
 	{\Large Bachvarov, Aleksandar }\\
 	{\Large Dimitrov, Atanas }\\
 	{\Large Mortazavi Moshkenan, Houraalsadat }\\
 	{\Large Sakly, Khalil }\\
 	{\Large Slobodyanik, Anastasia }\\
 	{\Large Voneva, Sonya}\\
 	\vfill
 	{\large 14.03.18 \par}
 	\end{center}
 \end{titlepage}
 
 \tableofcontents
 \newpage
 \section{Einleitung}
Dieses Dokument beschreibt ...

 \newpage
 \section{Qualität}
 Wie wird Qualität gemessen.
 
 \newpage
\section{Komponenten-Test}
Views unit Tests

\begin{longtable}[c]{|l|l|l|l|l|}
\caption{My caption}
\label{my-label}\\
\hline
\textbf{\#} & \textbf{Testcase}                 & \textbf{Beschreibung} & \textbf{Erwartung} & \textbf{Ergebnis} \\ \hline
\endfirsthead
%
\endhead
%

1 & TestHomeView 
& todo %description
& todo %expected
& todo	%result
\\ \hline

2 & TestResourceManager
& todo %description
& todo %expected
& todo	%result
\\ \hline

3 & TestUserManager
& todo %description
& todo %expected
& todo	%result
\\ \hline

4 & TestProfileView
& todo %description
& todo %expected
& todo	%result
\\ \hline

5 & TestMyResourcesView
& todo %description
& todo %expected
& todo	%result
\\ \hline

6 & TestSendDeletionRequest
& todo %description
& todo %expected
& todo	%result
\\ \hline

7 & TestCancelDeletionRequest 
& todo %description
& todo %expected
& todo	%result
\\ \hline

8 & TestApproveAccesRequest 
& todo %description
& todo %expected
& todo	%result
\\ \hline

9 & TestDenyAccesRequest
& todo %description
& todo %expected
& todo	%result
\\ \hline

10 & TestSendAccessRequest
& todo %description
& todo %expected
& todo	%result
\\ \hline

11 & TestCancelAccessRequest 
& todo %description
& todo %expected
& todo	%result
\\ \hline

12 & TestDeleteResourceView 
& todo %description
& todo %expected
& todo	%result
\\ \hline

13 & TestDeleteResourceView 
& todo %description
& todo %expected
& todo	%result
\\ \hline

14 & TestEditNameView
& todo %description
& todo %expected
& todo	%result
\\ \hline

15 & TestResourcesOverview
& todo %description
& todo %expected
& todo	%result
\\ \hline

16 & TestResourcesOverviewSearch
& todo %description
& todo %expected
& todo	%result
\\ \hline

17 & TestPermissionEditingView
& todo %description
& todo %expected
& todo	%result
\\ \hline

18 & TestPermissionEditingViewSearch 
& todo %description
& todo %expected
& todo	%result
\\ \hline
\end{longtable}
\newpage


\section{Integration-Test}
Test des Zusammenspiels der Komponenten.

\newpage
\section{System-Test}
Manuelle Tests.

\newpage
\section{Zusammenfassung}
Testüberdeckung. Fazit über die Qualität des Produkts.
\end{document}
\grid
