\documentclass[parskip=full,11pt]{scrartcl}
%\usepackage{pdfpages}
\usepackage[utf8]{inputenc}
\usepackage{amssymb}
\usepackage[T1]{fontenc}
\usepackage[german]{babel}
\usepackage[yyyymmdd]{datetime} 
\usepackage{hyperref}
\usepackage[toc, nonumberlist, automake]{} %added automake option
\usepackage{csquotes}
\usepackage{graphicx}
\hypersetup{
 		pdftitle={Implementierung},
 }
\usepackage{fancyhdr}%<-------------to control headers and footers
\usepackage[a4paper,margin=1in,footskip=.25in]{geometry}
\fancyhf{}
\fancyfoot[C]{\thepage} %<----to get page number below text
\pagestyle{fancy} %<-------the page style itself
 
\title{Implementierung}
\subtitle{Autorisierungsmanagement für eine virtuelle Forschungsumgebung für Geodaten}
\author{Alex\\Anastasia\\Atanas\\Dannie\\ Houra\\Sonya\\}
\date{11.01.18}
 % define custom lists
\usepackage{enumitem}


\usepackage{linegoal,listings}
\newsavebox{\mylisting}
\makeatletter
\newcommand{\lstInline}[2][,]{%
	\begingroup%
	\lstset{#1}% Set any keys locally
	\begin{lrbox}{\mylisting}\lstinline!#2!\end{lrbox}% Store listing in \mylisting
	\setlength{\@tempdima}{\linegoal}% Space left on line.
	\ifdim\wd\mylisting>\@tempdima\hfill\\\fi% Insert line break
	\lstinline!#2!% Reset listing
	\endgroup%
}
\makeatother
\setlength{\parindent}{0pt}% Just for this example

\lstset{basicstyle=\footnotesize\ttfamily,breaklines=true}
\lstset{framextopmargin=50pt,frame=bottomline,showstringspaces=false,upquote=true}


 
\begin{document}
 
 \begin{titlepage}
 	
 	\begin{center}
 	\includegraphics[width=0.5\linewidth]{res/KITLogo.png}\\
 	\vspace{2cm}
 	{\scshape\LARGE\bfseries Implementierung \par}
 	\vspace{0.5cm}
 	{\scshape\Large Praxis der Softwareentwicklung\\}
 	\vspace{1cm}
 	{\scshape\Large Wintersemester 17/18\\}
 	\vspace{2cm}
 	{\huge\bfseries Autorisierungsmanagement für eine virtuelle Forschungsumgebung für Geodaten\par}
 	\vspace{2cm}
 	\vfill
 	{\bfseries {\Large Autoren}:\par}
 	{\Large Bachvarov, Aleksandar }\\
 	{\Large Dimitrov, Atanas }\\
 	{\Large Mortazavi Moshkenan, Houraalsadat }\\
 	{\Large Sakly, Khalil }\\
 	{\Large Slobodyanik, Anastasia }\\
 	{\Large Voneva, Sonya}\\
 	\vfill
 	{\large 07.02.18 \par}
 	\end{center}
 \end{titlepage}
 
 \tableofcontents
 \newpage
 \section{Einleitung}
Dieses Dokument beschreibt die Änderungen, welche an unserem Entwurfdokument vorgenommen wurden, um die Funktionalität des Projekts ``Autorisierungsmanagement für eine virtuelle Forschungsumgebung für Geodaten`` zu gewährleisten. Das Ziel dieses Dokuments ist, die Gründe für diese Änderungen zu erläutern, beziehungsweise Probleme aufzuzeigen, die sich während der Implementierung ergeben haben.\\\\
Es werden Änderungen an Datenhaltung und Applikationslogik beschrieben sowie begründet. Des Weiteren wird ein Überblick darüber vermittelt, welche Musskriterien implementiert wurden, welche nicht und warum. Die implementierten Wunschkriterien werden ebenfalls kurz genannt.\\\\
Am Schluss des Dokuments befindet sich ein Vergleich zwischen unserem ursprünglichen und tatsächlichen Implementierungsplan sowie Erläuterungen eventueller Verzögerungen im Zeitplan.

 \newpage
 \section{Änderungen am Entwurf}
 
 \subsection{Model}
Während der Designphase wurden Klassen des Model-Pakets entsprechend der Prinzipien objektorientierter Programmierung entworfen. Während der Einarbeitung in das Django-Framework hat sich allerdings herausgestellt, dass Model-Klassen eher als Datenstruktur-Träger benutzt werden und keine Anwendungslogik in sich kapseln. Daher wurde sämtliche Funktionalität in dem View-Paket implementiert, sodass Model-Klassen ausschließlich die Datenbankstruktur definieren.
 
\begin{itemize}
\item \textbf{Klasse CustomUser}\\
Klasse \textit{CustomUser} erbt von der Django-Klasse \textit{User} und ersetzt Klasse \textit{User} aus dem Entwurfdokument. Sämtliche Attribute werden von Django vordefiniert und mussten nicht zusätzlich implementiert werden.

\item \textbf{Klasse Admin}\\
Da diese Klasse keine Funktionalität in sich kapselt, hat sich herausgestellt, dass sie durch die Benutzung des Attributs \textit{is staff} ersetzt werden kann. Dieses Attribut wird in der \textit{User}-Klasse von Django vordefiniert.
 
\item\textbf{Klasse Resource}\\
Um Lese- und Besitzerrechte zu implementieren, werden der Klasse zusätzliche Attribute hinzugefügt: die Listen \textit{Readers} und \textit{Owners}.

\item \textbf{Klasse ResourceType}\\
TO DO

\item \textbf{Klasse Request}\\
Dieser Klasse wurde das \textit{Description}-Attribut hinzugefügt. Dieses Attribut hat Type \textit{String} und dient dazu, die Begründung zu speichern, welche vom Absender bei der Erstellung des Requests eventuell eingegeben wurde.

\item \textbf{Klassen Logging und EmailMessages}\\
Diese von Django vordefinierten Klassen mussten nicht extra implementiert werden. 
\end{itemize}

\newpage
\subsection{View} 
 
\begin{itemize}
\item \textbf{Klasse ChosenRequestsView}\\
Funktionalität dieser Klasse wurde auf vier Views verteilt:
\begin{itemize}
\item \textit{ApproveAccessRequest}
\item \textit{DenyAccessRequest}
\item \textit{ApproveDeletionRequest}
\item \textit{DenyDeletionRequest}
\end{itemize}
Diese Unterteilung hat bessere Trennung der Anwendungslogik für Bearbeitung unterschiedlicher Requests zur Folge. 

\item \textbf{Klasse DeleteResourceView}\\
Dieser View erweitert Funktionalität von Klassen \textit{ManageResourcesView} und \textit{ResourcesOverview} und wird zum Löschen der Ressourcen von Administratoren des Portals benutzt. Absenden eines Löschrequests wird stattdessen im \textit{SendDeletionRequestView} implementiert. 

\item \textbf{Klassen PermissionForChosenResourceView,  PermissionsForResourceView und PermissionsForUsersView}\\
Diese Views wurden durch \textit{PermissionEditingView} ersetzt und für bessere Übersichtlichkeit durch  \textit{PermissionEditingViewSearch} erweitert. Bearbeitung der Rechte aller Ressourcen in einem View hat bessere Benutzbarkeit des Portals zur Folge.

\item \textbf{Klassen ManageUsersView und ManageResourcesView}\\
Diese Views wurden durch vordefinierte Django-Funktionalität ersetzt.

\item \textbf{Klasse ResourcesOverview}\\
Dieser View wurde durch einen zusätzlichen View \textit{ResourcesOverviewSearch} für bessere Benutzbarkeit des Portals erweitert.

\item \textbf{Klasse RequestView}\\
Funktionalität dieses Views wurde für bessere Struktur der Anwendungslogik auf vier Views verteilt:
\begin{itemize}
\item \textit{SendAccessRequest}
\item \textit{CancelAccessRequest}
\item \textit{ApproveDeletionRequest}
\item \textit{DenyDeletionRequest}
\end{itemize}

\item \textbf{Klasse ResourceInfoView}\\
Funktionalität dieses Views wurde durch einen Modaldialog implementiert, um überflüssige Codezeilen zu sparen.
\newpage
\item \textbf{Zusätzlich implementierte Views}\\
Während der Implementierung hat sich herausgestellt, dass einige Funktionalitäten des Portals zusätzliche View-Klassen benötigen. Die noch nicht erwähnten Views heißen:
\begin{itemize}
\item \textit{AddNewResourceView} dient zur Erstellung einer neuen Ressource;
\item \textit{EditNameView} wird zum Ändern des Benutzernamens benutzt.
\end{itemize} 
\end{itemize}

\subsection{URL-Verzeichnis}
Änderungen an Views haben dementsprechende Änderungen am URL-Verzeichnis verursacht. Eine URL lokalisiert einen View, der entweder als ein Main-Fenster oder dessen Teil präsentiert wird.  \\
\renewcommand{\labelitemi}{$\bullet$}
\renewcommand{\labelitemii}{$\bullet$}
\renewcommand{\labelitemiii}{$\bullet$} 
 
\begin{itemize}[itemsep=0pt]

\item \textbf{/home} – Mockup-Seite. Subseiten:
	\begin{itemize}[itemsep=0pt]
	\item \textbf{/register}
	\item \textbf{/login}
	\item \textbf{/logout}
	\end{itemize}

\item \textbf{/resource-manager}
\item \textbf{/user-manager}

\item \textbf{/profile}//
	Subseiten:
	\begin{itemize}[itemsep=0pt]
	\item \textbf{/my-resources}
		\begin{itemize}
		\item \textbf{/resourceid-edit-users-permissions}
		\item \textbf{/searchs}
		\end{itemize}

\end{itemize}



\item \textbf{/manage{\_}users} - Administratorseite zur Benutzerverwaltung, enthält Liste von Benutzern und Subseiten:
\begin{itemize}[itemsep=0pt]
\item \textbf{/block{\_}user} - Dialog zum Blockieren des Users.
\item \textbf{/delete{\_}user} - Dialog zum Löschen des Users.
\item \textbf{/$<$userID$>${\_}permissions{\_}for{\_}resources} - Dialog zur Änderung von Rechten des Benutzers.
\end{itemize}

\item \textbf{/manage{\_}resources} - Administratorseite für Ressourcenverwaltung, enthält Liste von Ressourcen und Subseiten:
\begin{itemize}[itemsep=0pt]
\item \textbf{/$<$resourceID$>${\_}permissions{\_}for{\_}users} - Dialog zur Änderung von Zugriffsrechten für die Ressource. 
\item \textbf{/delete{\_}resource} - Dialog zum Löschen der Ressource.
\end{itemize}
\item \textbf{/resources{\_}owerview} - Übersicht von Ressourcen, enthält Liste aller Ressourcen und folgende Subseiten:
\begin{itemize}[itemsep=0pt]
\item \textbf{/$<$resourceID$>${\_}info} - enthält Metadaten der Ressource.
\item \textbf{/$<$resourceID$>${\_}send{\_}request} - Dialog zum Absenden des Requests für die Ressource.
\end{itemize}
\item \textbf{/<resourceID>} - Ressourcenseite, enthält Metadaten und Link für Zugriff auf die Ressource.
\end{itemize} 
\newpage 
\subsection{Datenbank}
 \begin{figure}[ht!]
 	\centering
 	\includegraphics[width=0.9\textwidth]{res/database.png}
 	\caption{Schema der Datenbank.}
 \end{figure}

\newpage
\section{Implementierte Kriterien}
\subsection{Musskriterien}
Alle Musskriterien wurden implementiert.

\subsection{Wunschkriterien}
Folgende Wunschkriterien wurden implementiert:
\begin{itemize}
\item Wird eine Ressource gelöscht, so werden alle Besitzer per E-Mail benachrichtigt.
\item Der Administrator kann einen Benutzer blockieren.
\item Der Administrator kann Benutzer anhand von Vorname und/oder Nachname suchen.
\end{itemize}


\newpage
\section{Unittests}


\newpage
\section{Implementierungsplan}
 \begin{figure}[ht!]
 	\centering
 	\includegraphics[width=0.9\textwidth]{res/gannt_plan.png}
 	\caption{Ursprünglicher Implementierungsplan.}
 \end{figure}
  \begin{figure}[ht!]
 	\centering
 	\includegraphics[width=0.9\textwidth]{res/gannt_real.png}
 	\caption{Reale Zeitablauf der Implementierung.}
 \end{figure}
\end{document}
\grid
