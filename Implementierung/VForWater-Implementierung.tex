\documentclass[parskip=full,11pt]{scrartcl}
%\usepackage{pdfpages}
\usepackage[utf8]{inputenc}
\usepackage{amssymb}
\usepackage[T1]{fontenc}
\usepackage[german]{babel}
\usepackage[yyyymmdd]{datetime} 
\usepackage{hyperref}
\usepackage[toc, nonumberlist, automake]{glossaries} %added automake option
\usepackage{csquotes}
\usepackage{graphicx}
\hypersetup{
 		pdftitle={Implementierung},
 		bookmarks=true,
 }
\usepackage{fancyhdr}%<-------------to control headers and footers
\usepackage[a4paper,margin=1in,footskip=.25in]{geometry}
\fancyhf{}
\fancyfoot[C]{\thepage} %<----to get page number below text
\pagestyle{fancy} %<-------the page style itself
 
\title{Implementierung}
\subtitle{Autorisierungsmanagement für eine virtuelle Forschungsumgebung für Geodaten}
\author{Alex\\Anastasia\\Atanas\\Dannie\\ Houra\\Sonya\\}
\date{11.01.18}
 % define custom lists
\usepackage{enumitem}

% add glossary
\makeglossaries
\newglossaryentry{Geheimnisprinzip}
{
	name={Geheimnisprinzip},
	description={Vom Innenleben einer Klasse soll der Verwender – gemeint sind sowohl die Algorithmen, die mit der Klasse arbeiten, als auch der Programmierer, der diese entwickelt – möglichst wenig wissen müssen},
}


\usepackage{linegoal,listings}
\newsavebox{\mylisting}
\makeatletter
\newcommand{\lstInline}[2][,]{%
	\begingroup%
	\lstset{#1}% Set any keys locally
	\begin{lrbox}{\mylisting}\lstinline!#2!\end{lrbox}% Store listing in \mylisting
	\setlength{\@tempdima}{\linegoal}% Space left on line.
	\ifdim\wd\mylisting>\@tempdima\hfill\\\fi% Insert line break
	\lstinline!#2!% Reset listing
	\endgroup%
}
\makeatother
\setlength{\parindent}{0pt}% Just for this example

\lstset{basicstyle=\footnotesize\ttfamily,breaklines=true}
\lstset{framextopmargin=50pt,frame=bottomline,showstringspaces=false,upquote=true}


 
\begin{document}
 
 \begin{titlepage}
 	
 	\begin{center}
 	\includegraphics[width=0.5\linewidth]{res/KITLogo.png}\\
 	\vspace{2cm}
 	{\scshape\LARGE\bfseries Implementierung \par}
 	\vspace{0.5cm}
 	{\scshape\Large Praxis der Softwareentwicklung\\}
 	\vspace{1cm}
 	{\scshape\Large Wintersemester 17/18\\}
 	\vspace{2cm}
 	{\huge\bfseries Autorisierungsmanagement für eine virtuelle Forschungsumgebung für Geodaten\par}
 	\vspace{2cm}
 	\vfill
 	{\bfseries {\Large Autoren}:\par}
 	{\Large Bachvarov, Aleksandar }\\
 	{\Large Dimitrov, Atanas }\\
 	{\Large Mortazavi Moshkenan, Houraalsadat }\\
 	{\Large Sakly, Khalil }\\
 	{\Large Slobodyanik, Anastasia }\\
 	{\Large Voneva, Sonya}\\
 	\vfill
 	{\large 07.02.18 \par}
 	\end{center}
 \end{titlepage}
 
 \tableofcontents
 
 \newpage
 \section{Einleitung}
 
Anschluss auf Pflichtenheft und Entwurf

 
 \section{Änderungen}
 
 \subsection{Model}
Änderungen im Model\\
 
 \subsection{View} %todo erweitern/korrigieren wenn möglich
Änderungen im view\\

 \subsection{URL Verzeichnis}
Änderungen in URLs\\
 
\subsection{Datenbank}
Änderungen in DB Struktur\\


 \section{Implementierte Kriterien}
 \subsection{Musskriterien}
 
 \subsection{Wunschkriterien}
 
 
 \section{Unittests}
\section{Implementierungsplan}
 \newpage
\setglossarystyle{altlist}
\printglossary	
	
 \end{document}
\grid
